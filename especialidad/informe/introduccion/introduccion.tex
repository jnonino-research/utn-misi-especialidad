%%%%%%%%%%%%%%%%%%%%%%%%%%%%%%%%%%%%%%%%%%%%%%%%%%%%%%%%%%%%%%%%%%%%%%%%%%%%%%%%%
%																				%
%	TRABAJO:	Trabajo Final													%
%				Especialidad en Ingenier�a en Sistemas de Informaci�n			%
%																				%
%		Titulo:	Procesamiento de Datos en Tiempo Real							%
%																				%
%		Autor:	Juli�n Nonino													%
%																				%
%	Introducci�n																%	
%																				%
%	A�o: 2016																	%
%																				%
%%%%%%%%%%%%%%%%%%%%%%%%%%%%%%%%%%%%%%%%%%%%%%%%%%%%%%%%%%%%%%%%%%%%%%%%%%%%%%%%%

\chapter{Introducci�n}
	
En los �ltimos a�os, con las llegada de las Redes Sociales, Big Data, Internet
de las Cosas, entre otras, la cantidad de datos generados creci�
exponencialmente. Paralelamente, se vi� incrementada la necesidad de tomar
acciones en base a dichos datos en el menor tiempo posible. En medio de �ste
fen�meno, surgen los sistemas en la nube, con escalabilidad para crecer y
decrecer en base a los requerimientos de procesamiento de cada momento.
Tecnolog�as como \emph{Docker} aparecen como alternativas a las antiguas maneras
de desarrollar y desplegar sistemas y servidores, acercandonos m�s a la
\emph{Infrastructura como C�digo}. Surgen tambi�n tecnolog�as como \emph{Apache
kafka} y \emph{Apache Storm} para transmitir mensajes y procesar dicho mensajes
respectivamente. \emph{Apache Zookeeper} aparece tambi�n como una herramienta
robusta para sincronizar y coordinar servicios.

�ste trabajo surge con la idea de utilizar dichas tecnolog�as para ejemplificar
su uso como parte de un sistema.
	
\section{Objetivos}

	El objetivo principal de �ste trabajo es implementar un sistema de procesamiento
	de datos en tiempo real como prueba de concepto utilizando las �ltimas
	tecnolog�as de la industria como son Docker, Apache Zookeeper, Apache Kafka y
	Apache Storm.
	
	\subsection{Objetivos Secundarios}
	
		\begin{itemize}
		    \item Plantear un modelo de generaci�n y procesamiento de datos sencillo que
		    ayude a visualizar el funcionamiento del sistema.
		    \item Demostrar el rol de Apache Zookeeper dentro de los sistemas
		    distribuidos. Mostrar su implementaci�n a trav�s de una prueba de concepto.
		    \item Demostrar el rol de Apache Kafka y Apache Storm dentro de los sistemas
		    de procesamiento de datos en tiempo real.
		    \item Estudiar y utilizar Docker como herramienta de despliegue de los
		    componentes del sistema ayudando a la escalabilidad del sistema.
		    \item Utilizar Docker Compose como mecanismo de despliegue del
		    sistema y conexi�n de los componentes del mismo.
		\end{itemize}

\section{Manejo de las Configuraciones}

	\subsection{Ubicaci�n del Proyecto}
		
		Los archivos del proyecto, incluyendo c�digo fuente, documentos, archivos
		LaTeX del informe del trabajo y otros recursos, se encuentran en un proyecto
		privado en la herramienta \emph{BitBucket} en la direcci�n
		\url{bitbucket.org/jnonino/especialidad-utn}.
		
		Dado que es un repositorio privado, se debe contar con un usuario de la
		herramienta y se deben otorgar permisos de acceso.
		
	\subsection{Herramientas y Frameworks Utilizados}
		
		Para el desarrollo de este trabajo, se utilizan las siguientes herramientas.

		\begin{itemize}
			\item \emph{Git} como herramienta de control de versiones.
			\item \emph{BitBucket} como repositorio Git.
			\item \emph{Eclipse Neon.1} como IDE para la escritura del informe en LaTeX.
			\item \emph{IntelliJ IDEA Ultimate 2016.3} para el desarrollo de las im�genes
			de Docker y del codigo Java de las aplicaciones desarrolladas.
			\item \emph{Maven} como herramienta de automatizaci�n de compilaci�n.
			\item \emph{Java Development Kit 7} como lenguaje de programaci�n y
			herramientas de compilaci�n y ejecuci�n.
		\end{itemize}

		Entre las piezas de software y/o frameworks utilizados como parte del sistema
		desarrollado, se encuentran:
		
		\begin{itemize}
			\item \emph{Docker} como herramienta de generaci�n de contenedores
			independientes para el despliegue del sistema.
			\item \emph{Docker Compose} como herramienta de orquestaci�n y coordinaci�n
			entre las diferentes im�genes de Docker que deben correr.
			\item \emph{Apache Zookeeper} como herramienta de coordinaci�n de servicios
			entre los componentes del sistema.
			\item \emph{Apache Kafka 0.9.0.1} como sistema de recepci�n y env�o de los
			datos que arriban al sistema.
			\item \emph{Apache Storm 0.9.7} como sistema de procesamiento de datos.
		\end{itemize}
