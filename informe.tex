%%%%%%%%%%%%%%%%%%%%%%%%%%%%%%%%%%%%%%%%%%%%%%%%%%%%%%%%%%%%%%%%%%%%%%%%%%%%%%%%%
%																				%
%	TRABAJO:	Trabajo Final													%
%				Especialidad en Ingenier�a en Sistemas de Informaci�n			%
%																				%
%		Titulo:																	%
%																				%
%		Autor:	Juli�n Nonino													%
%																				%
%	DOCUMENTO PRINCIPAL															%	
%																				%
%	A�o: 2016																	%
%																				%
%%%%%%%%%%%%%%%%%%%%%%%%%%%%%%%%%%%%%%%%%%%%%%%%%%%%%%%%%%%%%%%%%%%%%%%%%%%%%%%%%

\documentclass[a4paper,12pt,openright,twoside]{book}

% Paquetes
	% Idioma y codificacion de caracteres
		\usepackage[spanish]{babel}
		\usepackage[latin1]{inputenc}
	% Figuras
		\usepackage{graphicx}
		\usepackage{subfigure}
		\usepackage{float} % Para posicionar im�genes donde uno quiera. Solo hay que poner la opcion [H]
	% Apendice
		\usepackage{appendix}
	%Tablas	
	%\usepackage{tabular}
	% Margenes
		\usepackage{anysize}
	% Tabla de conteido
		\usepackage[tight]{shorttoc}
	% Matematica
		\usepackage[cmex10]{amsmath}
		\usepackage{amssymb}
	% Colores
		\usepackage{color}
		% Definicion de colores
			\definecolor{dkgreen}{rgb}{0,0.6,0}
			\definecolor{gray}{rgb}{0.5,0.5,0.5}
			\definecolor{mauve}{rgb}{0.58,0,0.82}
			\definecolor{violeta}{RGB}{127,0,85}
	% Insertar c�digo
		\usepackage{listings}
		\lstdefinestyle{bash}{
   			language=Bash,
   			showspaces=false,
   			showstringspaces=false,
   			basicstyle=\ttfamily,
   			columns=flexible,
   			stringstyle=\color{javastring},
   			keywordstyle=\color{violeta}\ttfamily\textbf,
   			commentstyle=\color{dkgreen}\ttfamily\textit
 		}
	% URLs
		\usepackage{url}
	% Referencias
		%\usepackage[dcucite]{harvard}
		\usepackage{hyperref}
		\usepackage[style=numeric,backend=bibtex,sorting=none]{biblatex}
		\addbibresource{./informe/referencias.bib}

% Margenes
	% Controla los m�rgenes {izquierda}{derecha}{arriba}{abajo}
		\marginsize{3cm}{3cm}{2.5cm}{2.5cm}

% Encabezados
	\pagestyle{headings}
		
% Documento
\begin{document}
 
	% Reeescritura de comandos
		\renewcommand{\lstlistingname}{C�digo}
		\renewcommand{\appendixname}{Ap�ndice}
		\renewcommand{\appendixtocname}{Ap�ndice}
		\renewcommand{\tablename}{\textbf{Tabla}} 	% Para poner la palabra en mayusucula
		\renewcommand{\figurename}{\textbf{Figura}} % Para poner la palabra en mayuscula
		\renewcommand{\contentsname}{�ndice}
		\renewcommand{\listtablename}{�ndice de tablas}
		\renewcommand{\listfigurename}{�ndice de Figuras}

		\setcounter{secnumdepth}{3} % Para numerar subsubsecciones
		\setcounter{tocdepth}{3}	% Para incluir subsubsecciones en la TOC	
	% INICIO DE LA PRIMERA PARTE. Resumen ejecutivo
 		\frontmatter
 		% Portada
 			\begin{titlepage}
 				%%%%%%%%%%%%%%%%%%%%%%%%%%%%%%%%%%%%%%%%%%%%%%%%%%%%%%%%%%%%%%%%%%%%%%%%%%%%%%%%%
%																				%
%	TRABAJO:	Trabajo Final													%
%				Especialidad en Ingenier�a en Sistemas de Informaci�n			%
%																				%
%		Titulo:																	%
%																				%
%		Autores:	Julian Nonino												%
%																				%
%	Portada																		%	
%																				%
%	A�o: 2016																	%
%																				%
%%%%%%%%%%%%%%%%%%%%%%%%%%%%%%%%%%%%%%%%%%%%%%%%%%%%%%%%%%%%%%%%%%%%%%%%%%%%%%%%%

% PAGINA ANTERIOR
	\vspace*{0.15in}

	\begin{center}		
	
		\begin{LARGE}
			\textbf{Procesamiento de Datos en Tiempo Real}
		\end{LARGE}
		\\
		\vspace*{0.15cm}
		\begin{Large}
			\textbf{Conceptos y An�lisis de Herramientas}
		\end{Large}
		
		\vspace*{0.15cm}
		\rule{15cm}{0.1mm} 
		\vspace*{0.15cm}
		\begin{Large}Especializaci�n en Ingenier�a en Sistemas de
		Informaci�n\end{Large}\\
	
		\vspace*{2cm}
	
		\begin{Huge}
			Julian Nonino
		\end{Huge}
		\\
		
		\vspace*{2cm}
		
%		\begin{Large}
%			Director
%			\\
%			\vspace*{0.5cm}
%			Mart�n Miceli
%		\end{Large}
		
		\vspace*{4.5cm}
		
		\begin{figure}[H]
			\centering
			\includegraphics[width=.10\linewidth]{./portada/logo_utn}
		\end{figure}	
		
		
		\begin{large}Universidad Tecnol�gica Nacional\end{large}\\
		\vspace*{0.25cm}
		\begin{large}Facultad Regional C�rdoba\end{large}\\
		\vspace*{0.25cm}
		\begin{large}Direcci�n de Posgrado\end{large}\\
		\vspace*{0.5cm}
		C�rdoba\\
		- 2016 -
	\end{center}

% PAGINA POSTERIOR
	\newpage
	\mbox{}
	\thispagestyle{empty}
	
 				\thispagestyle{empty}
 			\end{titlepage}
 		% Resumen
 			% Resumen ejecutivo
 				%%%%%%%%%%%%%%%%%%%%%%%%%%%%%%%%%%%%%%%%%%%%%%%%%%%%%%%%%%%%%%%%%%%%%%%%%%%%%%%%%
%																				%
%	TRABAJO:	Trabajo Final													%
%				Especialidad en Ingenier�a en Sistemas de Informaci�n			%
%																				%
%		Titulo:																	%
%																				%
%		Autor:	Juli�n Nonino													%
%																				%
%	Resumen Ejecutivo															%	
%																				%
%	A�o: 2016																	%
%																				%
%%%%%%%%%%%%%%%%%%%%%%%%%%%%%%%%%%%%%%%%%%%%%%%%%%%%%%%%%%%%%%%%%%%%%%%%%%%%%%%%%

\chapter*{Resumen Ejecutivo}

 			% Executive Summary
 				%%%%%%%%%%%%%%%%%%%%%%%%%%%%%%%%%%%%%%%%%%%%%%%%%%%%%%%%%%%%%%%%%%%%%%%%%%%%%%%%%
%																				%
%	TRABAJO:	Trabajo Final													%
%				Especialidad en Ingenier�a en Sistemas de Informaci�n			%
%																				%
%		Titulo:	Procesamiento de Datos en Tiempo Real							%
%																				%
%		Autor:	Juli�n Nonino													%
%																				%
%	Executive Summary															%	
%																				%
%	A�o: 2016																	%
%																				%
%%%%%%%%%%%%%%%%%%%%%%%%%%%%%%%%%%%%%%%%%%%%%%%%%%%%%%%%%%%%%%%%%%%%%%%%%%%%%%%%%

\chapter*{Executive Summary}

 		% Tabla de contenido inicial
 			\shorttableofcontents{Tabla de contenido}{0}
 		
 	%INICIO DEL TEXTO PRINCIPAL
 		\mainmatter
 		% Introducci�n
 			\part{Introducci�n}
 				% Introducci�n
					%%%%%%%%%%%%%%%%%%%%%%%%%%%%%%%%%%%%%%%%%%%%%%%%%%%%%%%%%%%%%%%%%%%%%%%%%%%%%%%%%
%																				%
%	TRABAJO:	Trabajo Final													%
%				Especialidad en Ingenier�a en Sistemas de Informaci�n			%
%																				%
%		Titulo:																	%
%																				%
%		Autor:	Juli�n Nonino													%
%																				%
%	Introducci�n																%	
%																				%
%	A�o: 2016																	%
%																				%
%%%%%%%%%%%%%%%%%%%%%%%%%%%%%%%%%%%%%%%%%%%%%%%%%%%%%%%%%%%%%%%%%%%%%%%%%%%%%%%%%

\chapter{Introducci�n}
	

	
\section{Objetivos}

	El objetivo principal de �ste trabajo es implementar un sistema de procesamiento
	de datos en tiempo real como prueba de concepto utilizando las �ltimas
	tecnolog�as de la industria como son Docker, Apache Zookeeper, Apache Kafka y
	Apache Storm.
	
	\subsection{Objetivos Secundarios}
	
		\begin{itemize}
		    \item Plantear un modelo de generaci�n y procesamiento de datos sencillo que
		    ayude a visualizar el funcionamiento del sistema.
		    \item Demostrar el rol de Apache Zookeeper dentro de los sistemas
		    distribuidos. Mostrar su implementaci�n a trav�s de una prueba de concepto.
		    \item Demostrar el rol de Apache Kafka y Apache Storm dentro de los sistemas
		    de procesamiento de datos en tiempo real.
		    \item Estudiar y utilizar Docker como herramienta de despliegue de los
		    componentes del sistema ayudando a la escalabilidad del sistema.
		    \item Utilizar Docker Compose como mecanismo de despliegue del
		    sistema y conexi�n de los componentes del mismo.
		\end{itemize}

				% Marco Te�rico
					%%%%%%%%%%%%%%%%%%%%%%%%%%%%%%%%%%%%%%%%%%%%%%%%%%%%%%%%%%%%%%%%%%%%%%%%%%%%%%%%%
%																				%
%	TRABAJO:	Trabajo Final													%
%				Especialidad en Ingenier�a en Sistemas de Informaci�n			%
%																				%
%		Titulo:																	%
%																				%
%		Autores:	Julian Nonino												%
%																				%
%	Marco Te�rico																%	
%																				%
%	A�o: 2016																	%
%																				%
%%%%%%%%%%%%%%%%%%%%%%%%%%%%%%%%%%%%%%%%%%%%%%%%%%%%%%%%%%%%%%%%%%%%%%%%%%%%%%%%%

\chapter{Marco Te�rico}
\label{chapter_marco_teorico}

\section{Procesamiento de Datos en Tiempo Real}
\label{section_real_time}

	En los �ltimos tiempos, la demanda de procesamiento de flujos continuos de datos
	(data streams) se ha incrementado considerablemente. Esto se debe a que ya no es
	suficiente con procesar grandes vol�menes de datos. Los datos, adem�s, deben ser
	procesados r�pidamente permitiendo a los sistemas reaccionar ante los eventos lo
	antes posible. Ejemplos de sistemas que necesitan �ste nivel de procesamiento
	son los sistemas de detecci�n de fraude, monitoreo de recursos, comercio,
	etc�tera.

	\subsection{Big Data}
	
		El t�rmino Big Data, muy utilizado en la actualidad, hace referencia a lo que
		se conoce como las tres V, Volumen, Variedad y Velocidad. Con ello, se quiere
		indicar que un sistema Big Data no solo implica trabajar con grandes vol�menes
		de datos, sino que estos datos pueden ser muy variados y se deben procesar
		r�pidamente \cite{Wahner2014}.
	
		\begin{figure}[H]
			\centering
			\includegraphics[width=.5\linewidth]{./introduccion/img/real_time/big_data_tres_v}
		\end{figure}

	\subsection{Procesamiento de Flujos de Datos (Stream Processing)}
		
		Es un sistema dise�ado para analizar y actuar en tiempo real un flujo continuo
		de datos. En contraste a los modelos de procesamiento de datos tradicionales en
		los cuales los datos son primero almacenados y luego procesados y analizados,
		cuando se procesa un flujo de datos, los datos son procesados y analizados
		mientras entran en el sistema. Esto permite lo que se llama procesar datos en
		movimiento. Esto permite conectar a los procesadores de datos a fuentes de
		datos externas introduciendo a los mismos al flujo de procesamiento.
			
		Una soluci�n de procesamiento de datos en tiempo real debe ser capaz de:
		\begin{itemize}
		    \item Procesar cantidades enormes de datos permitiendo filtrado,
		    agregaci�n, predicci�n, alertas, reglas, etc�tera.
			\item Respuesta en tiempo real a los mensajes/eventos recibdos.
			\item Asegurar rendimiento y escalabilidad cuando el volumen de datos crece en
			tama�o y/o complejidad.
			\item Integraci�n f�cil y r�pida con la infraestructura y fuentes de datos
			existentes.
			\item R�pida implementaci�n y puesta en producci�n de nuevos requisitos de
			procesamiento.
		\end{itemize}
		
	\section{Docker}
	\label{section_docker}

	Docker es un proyecto de c�digo abierto que automatiza el despliegue de
	aplicaciones dentro de contenedores de software, proporcionando una capa
	adicional de abstracci�n y automatizaci�n de virtualizaci�n a nivel de sistema
	operativo en Linux. Docker utiliza caracter�sticas de aislamiento de recursos
	del kernel de Linux, tales como cgroups y espacios de nombres (namespaces) para
	permitir que \emph{contenedores} independientes se ejecuten dentro de una sola
	instancia de Linux, evitando la sobrecarga de iniciar y mantener m�quinas
	virtuales. \cite{WikipediaDocker}
	
	El soporte del kernel de Linux para los espacios de nombres a�sla de vista, en
	su mayor�a, una aplicaci�n del entorno operativo, incluyendo �rboles de proceso,
	red, ID de usuario y sistemas de archivos montados, mientras que los cgroups del
	kernel proporcionan aislamiento de recursos, incluyendo la CPU, la memoria, el
	bloque de E/S y de la red. Desde la versi�n 0.9, Docker incluye la librer�a
	libcontainer como su propia manera de utilizar directamente las facilidades de
	virtualizaci�n que ofrece el kernel de Linux, adem�s de utilizar las interfaces
	abstra�das de virtualizaci�n mediante libvirt, LXC (Linux Containers) y
	systemd-nspawn. \cite{WikipediaDocker}
	
	Docker implementa una API de alto nivel para proporcionar contenedores livianos
	que ejecutan procesos de manera aislada.
	Construido sobre las facilidades proporcionadas por el kernel de Linux
	(principalmente cgroups y namespaces), un contenedor Docker, a diferencia de una
	m�quina virtual, no requiere incluir un sistema operativo independiente. En su
	lugar, se basa en las funcionalidades del kernel y utiliza el aislamiento de
	recursos (CPU, la memoria, el bloque E/S, red, etc.) y namespaces separados
	para aislar de vista la aplicaci�n del sistema operativo. Docker accede a la
	virtualizaci�n del kernel de Linux ya sea directamente a trav�s de la biblioteca
	libcontainer (disponible desde Docker 0.9), o indirectamente a trav�s de
	libvirt, LXC o systemd-nspawn. \cite{WikipediaDocker}

	\begin{figure}[H]
		\centering
		\includegraphics[width=1\linewidth]{./introduccion/img/docker/vm_vs_docker}
		\caption{Contenedor Docker (derecha) versus M�quina Virtual (izquierda)
		\cite{WhatIsDocker2016}}
	\end{figure}

	Mediante el uso de contenedores, los recursos pueden ser aislados, los servicios
	restringidos, y se otorga a los procesos la capacidad de tener una visi�n casi
	completamente privada del sistema operativo con su propio identificador de
	espacio de proceso, la estructura del sistema de archivos, y las interfaces de
	red. Contenedores m�ltiples comparten el mismo n�cleo, pero cada contenedor
	puede ser restringido a utilizar s�lo una cantidad definida de recursos como
	CPU, memoria y E/S. \cite{WikipediaDocker}
	
	Usando Docker para crear y gestionar contenedores se puede simplificar la
	creaci�n de sistemas altamente distribuidos, permitiendo m�ltiples aplicaciones,
	las tareas de los trabajadores y otros procesos para funcionar de forma aut�noma
	en una �nica m�quina f�sica o en varias m�quinas virtuales. Esto permite que el
	despliegue de nodos se realice a medida que se dispone de recursos o cuando se
	necesiten m�s nodos, lo que permite una plataforma como servicio (PaaS -
	Plataform as a Service) de estilo de despliegue y ampliaci�n de los sistemas
	como Apache Cassandra, MongoDB o Riak. Docker tambi�n simplifica la creaci�n y
	el funcionamiento de las tareas de carga de trabajo o las colas y otros sistemas
	distribuidos. \cite{WikipediaDocker}

	\subsection{Imagenes y Contenedores\cite{GetStartedDocker2016}}
	
		Un \emph{contenedor} (container) es una version de un sistema operativo Linux,
		solo con los componentes m�s b�sicos. Una \emph{imagen} es software que se
		carga dentro del container al momento de ejecutar el comando \emph{run}.
	
\lstset{language=bash}
\begin{lstlisting}
docker run hello-world
\end{lstlisting}
	
		El comando \emph{run} recibe como par�metro requerido el nombre de la
		\emph{imagen} que se desea cargar en un \emph{contenedor}, en �ste caso
		\emph{hello-world}.
		
		Al correr dicho comando, Docker ejecuta las siguientes acciones:
		\begin{itemize}
		    \item Comprobar si existe en el sistema una imagen con el nombre
		    \emph{hello-world}.
		    \item En caso de que no exista dicha imagen en el sistema descargarla desde
		    el repositorio de im�genes configurado, por defecto es \emph{Docker Hub},
		    un repositorio propiedad de Docker donde existen miles de imagenes
		    disponibles. Es posible tener repositorios privados utilizando lo que se
		    conoce como \emph{Docker Registry}.
		    \item Cargar la imagen en el contenedor y ejecutarla.
		\end{itemize}
		
		Por otro lado, una imagen de Docker puede ejecutar un simple comando o cargar
		un complejo sistema de base de datos.
		
		Para construir una imagen de Docker, es necesario crear un archivo llamado
		\emph{Dockerfile}.
	
\lstset{language=bash}
\begin{lstlisting}
FROM ubuntu:16.04

RUN apt-get -y update

CMD["echo Hola"]
\end{lstlisting}
	
		El Dockerfile anterior buscara una imagen de Ubuntu con la etiqueta
		(\emph{tag}) \emph{16.04}. Luego ejecutar� un comando para actualizar los
		paquetes del sistema operativo y luego mostrar� el mensaje \emph{Hola}.
	
		El comando para construir una imagen de Docker es:

\lstset{language=bash}
\begin{lstlisting}
docker build -t miimagen .
\end{lstlisting}

		Se ejecutar� el comando \emph{build} para construir la imagen. El argumento
		\emph{-t} indica que se le pondr� la etiqueta \emph{miimagen} a la imagen y
		punto al final indica el directorio de contexto de la imagen, �sto es �til
		porque se pueden agregar archivos al container al momento de construir la
		imagen. En �ste caso, el contexto ser� el directorio donde se encuentra el
		Dockerfile.
	
		Luego, es posible cargar la imagen en un contenedor mediante el comando:
	
\lstset{language=bash}
\begin{lstlisting}
docker run miimagen
\end{lstlisting}

	\subsection{Crear nuevas etiquetas}
	
		Para ponerle una nueva etiqueta a una imagen, primero debemos encontrar el
		n�mero de identificaci�n de la imagen. �sto se hace corriendo el comando:

\lstset{language=bash}
\begin{lstlisting}
docker images
\end{lstlisting}

		El comando anterior, mostrar� una lista de las im�genes existentes en el
		sistema mostrando la �ltima etiqueta de la misma, el n�mero de identificaci�n,
		la fecha de creaci�n y el tama�o de la imagen.
		
		Luego, para aplicarle una nueva etiqueta, se ejecuta el comando:
	
\lstset{language=bash}
\begin{lstlisting}
docker tag <IMAGE_ID> <NUEVA_ETIQUETA>
\end{lstlisting}	
	
	\subsection{Docker Compose\cite{DockerComposeDocumentation}}
	
		Docker Compose es una herramienta que permite correr un sistema formado por
		m�ltiples contenedores. Para ello, se debe crear un archivo \emph{.yml} en el
		que se definan los servicios con los que va a contar la aplicaci�n. Cada
		servicio estar� formado por un contenedor corriendo una imagen de Docker.
		
		Para cada servicio pueden definirse nombres, puertos expuestos, conexiones de
		red, etc�tera, luego, con los siguientes comandos se puede operar con el
		sistema.
		
		Para una lista completa de los comandos de Docker Compose, acceder a
		\href{https://docs.docker.com/compose/reference/}{Docker Compose Command-Line
		Reference}\footnote{https://docs.docker.com/compose/reference/}.

	\subsection{Material}
	
		La informaci�n de �sta secci�n ha sido extra�da mayormente desde la
		documentaci�n de Docker\cite{GetStartedDocker2016} y Docker Compose \cite{DockerComposeDocumentation}.		
		
\section{Apache Kafka}
\label{section_apache_kafka}

	Kafka es un sistema de mensajes distribuido, particionado y con
	replicaci�n\cite{ApacheKafka090}.

	\begin{itemize}
	    \item Kafka mantiene los mensajes agrupados en categor�as llamadas
	    \emph{topics.}
		\item Los productores de mensajes se llaman \emph{producers}.
		\item Los consumidores de mensajes se llaman \emph{consumers}.
		\item Kafka corre en un cluster formado por uno o mas servidores. cada uno de
		ellos es llamado \emph{broker}.
	\end{itemize}
	
	\begin{figure}[H]
		\centering
		\includegraphics[width=.5\linewidth]{./introduccion/img/kafka/high_level_arch}
		\caption{Kafka, arquitectura de alto nivel\cite{ApacheKafka090}}
	\end{figure}

	\subsection{Topics}
	
		Los \emph{topics} de Kafka son categor�as de mensajes para los cuales Kafka
		mantiene registros particionados.
		
		Cada partici�n es una secuencia ordenada e inmutable de mensajes. El n�mero de
		orden de cada mensaje es llamado \emph{offset} e identifica univocamente a cada
		mensaje de la partici�n.
		
		Kafka mantiene los mensajes publicados por un per�odo de tiempo configurable,
		sin importar si fueron consumidos o no por alg�n proceso \emph{consumer}. Cada
		consumidor se encarga de mantener el \emph{offset} y tiene libertad para ir
		hacia atr�s y hacia adelante en los mensajes publicados para procesarlos.
		
		El tener los mensajes de un \emph{topic} particionados permite separar el
		\emph{topic} en varios servidores. �sto permite manejar grandes volumenes de
		datos y adem�s otorogar un nivel superior de paralelismo.
		
		\begin{figure}[H]
			\centering
			\includegraphics[width=.5\linewidth]{./introduccion/img/kafka/kafka_topics}
			\caption{Topics en Kafka\cite{ApacheKafka090}}
		\end{figure}
		
		Cada partici�n est� formada por un l�der (\emph{leader}) que se encuentra en
		uno de los servidores y por cero o m�s seguidores (\emph{followers}) que
		replican al l�der todo el tiempo en servidores distintos. Si el l�der falla,
		alguno de los seguidores se convertir� en el nuevo l�der garantizando que el
		sistema siga operando. la configuraci�n ideal es que cada servidor sea l�der de
		alguna partici�n y seguidor de las otras.
		
	\subsection{Productores}
		
		Los productores en Kafka son programas encargados de publicar datos en los
		\emph{topics}. El productor decide, para cada mensaje, el topic y la partici�n
		en el cual publicarlo. Generalmente la partici�n es elegida siguiendo un
		esquema \emph{round-robin} para lograr un �ptimo balance de carga entre
		particiones, pero se puede utilizar cualquier l�gica.
		
	\subsection{Consumidores}
	
		Los sitemas de mensajer�a pueden ser clasificados en dos categor�as,
		\emph{cola de mensajes} o \emph{publicaci�n-subscripci�n}. En el primero, los
		mensajes son encolados y cada mensaje es dirigido hacia alguno de los
		consumidores. En el segundo, cada mensaje es transmitido a todos los
		consumidores. kafka maneja ambos mundos con lo que se conoce como grupos de
		consumidores (\emph{consumer groups}).
		
		Cada consumidor debe ubicarse dentro de alguno de los grupos de consumidores y
		cuando un mensajes es publicado en un \emph{topic}, el mensaje es transmitido a
		un �nico consumidor de cada uno de los grupos de consumidores.
		
		\begin{figure}[H]
			\centering
			\includegraphics[width=.5\linewidth]{./introduccion/img/kafka/kafka_consumers}
			\caption{Grupos de Consumidores\cite{ApacheKafka090}}
		\end{figure}
		
		Si todos los consumidores se encuentran en el mismo grupo, el sistema funciona
		como una cola de mensajes distribuyendo la carga entre cada uno de los
		consumidores.
		
		Si todos los consumidores se encuentran en distintos grupos, el sistema
		funciona como un sistema publicaci�n-subscripci�n y todos los mensajes son
		transmitidos a todos los consumidores.
		
	\subsection{Material}
	
		La informaci�n de �ste cap�tulo ha sido extra�da mayormente desde la
		documentaci�n de Apache Kafka 0.9 \cite{ApacheKafka090}.
		
\section{Apache Storm}
\label{section_apache_storm}
	
	Apache Storm es una herramienta de procesamiento de datos en tiempo real de
	c�digo abierto y gratuita creada por Twitter y luego liberada en lo �rbita de
	los proyectos Apache.
	
	La finalidad de Storm es proveer un mecanismo confiable para procesamiento de
	flujos de datos ilimitados, haciendo para flujos de datos (\emph{realtime
	stream processing}) lo que Hadoop hace en procesamiento por lotes (\emph{batch
	processing})\cite{ApacheStorm101}.
	
	Deacuerdo a su documentaci�n, es capaz de procesar un mill�n de tuplas de datos
	por segundo por nodo. Provee caracter�sticas de escalabilidad, tolerancia a
	fallos, garant�as de que todos los datos ser�n procesados, etc�tera\cite{ApacheStorm101}.
	
	Una \emph{topolog�a} de Storm consume flujos de datos, los procesa y genera
	nuevos flujos de datos.
	
	\subsection{Conceptos B�sicos}
	
		En �sta secci�n se analizar�n los conceptos b�sicos que definen a un programa
		Storm.
		
		\begin{figure}[H]
			\centering
			\includegraphics[width=.9\linewidth]{./introduccion/img/storm/topology}
		\end{figure}
		
		\subsubsection{Topologies}
		
		Las \emph{topolog�as} son los contenedores de la l�gica de una aplicaci�n de
		procesamiento de datos en tiempo real. Es el an�logo a un trabajo de MapReduce
		de
		Hadoop\footnote{https://hadoop.apache.org/docs/current/hadoop-mapreduce-client/hadoop-mapreduce-client-core/MapReduceTutorial.html}.
		La diferencia principal con �stos �ltimos es que un trabajo MapReduce de
		Hadoop, eventualmente concluye mientras que las topolog�as pueden correr
		indefinidamente.
		Una topolog�a es un grafo formado por \emph{Spouts} y \emph{Bolts} conectados a
		trav�s de \emph{Stream Groupings}.
		
		\subsubsection{Streams}
		
			Los streams son una secuencia ilimitada de tuplas de datos que son creadas y
			procesadas de manera distribuida. Los streams se definen creando un esquema
			que contenga todos los campos de datos de cada tupla que forma parte del
			stream. Las tuplas pueden contener valores enteros (\emph{integer}), bytes,
			cadenas de caracteres (\emph{strings}), valores booleanos, etc�tera.
		
		\subsubsection{Spouts}
		
			Los spouts son la fuente de streams para la topolog�a. Generalemente leen
			tuplas de datos desde una fuente externa y las emiten dentro de la topolog�a
			para que sea procesada. Los spouts pueden ser:
			\begin{itemize}
			    \item \textbf{reliable} (confiable): es un spout capaz de reenviar una
			    tupla si Storm fall� en procesarla.
			    \item \textbf{unreliable} (no confiable): el spout \emph{se olvida} de las
			    tuplas en el momento en el que las emite hacia la topolog�a.
			\end{itemize}
		
		\subsubsection{Bolts}
	
			Dentro de una topolog�a, todo el procesamiento sobre los datos es realizado en
			los bolts. Los bolts pueden ser programados para realizar cualquier tarea como
			filtrado, agregaci�n, uniones con bases de datos, funciones, etc�tera.
			
			Los bolts reciben uno o varios streams de datos y pueden emitir nuevamente uno
			o varios de ellos.
	
		\subsubsection{Stream Grouping}
		
			Al definir una topolog�a, es necesaria especificar que streams debe recibir
			como entrada cada uno de los bolts. Los \emph{stream groupings} definen como
			los streams deben ser particionados en las tareas de cada bolt.
			
			Existen ocho \emph{stream groupings} predefinidos pero existe la posibilidad
			de crear nuevos implementando la interfaz \emph{CustomStreamGrouping}.
			\begin{itemize}
			    \item \emph{Shuffle grouping}: Las tuplas son distribuidas
			    aleatoriamente en las tareas de los bolts de manera tal que se
			    garantice que todos los bolts reciben las misma cantidad de tuplas.
			    
			    \item \emph{Fields grouping}: El stream es particionado deacuerdo a los
			    campos de datos que contenga la tupla. Por ejemplo, si la tupla contiene
			    un campo llamado \emph{usuario} y se agrupa el stream por el campo
			    \emph{usuario}, todos aquellas tuplas que tengan el mismo valor en dicho
			    campo ser�n procesadas por el mismo bolt.
			    
			    \item \emph{Partial Key grouping}: El stream es particionado de la misma
			    manera que en el caso de \emph{Fields grouping} solo que la carga es
			    balanceada entre dos bolts para proporcionar una mejor utilizaci�n de los
			    recursos.
			    
			    \item \emph{All grouping}: El stream de datos es replicado en TODAS las
			    tareas de los bolts.
			    
			    \item \emph{Global grouping}: El stream completo es dirigido hacia una
			    �nica tarea de un bolt.
			    
			    \item \emph{None grouping}: Al utilizar esta forma de agrupamiento, se
			    est� indicando que no es importante como el stream es dirigido hacia los
			    bolts.
			    
			    \item \emph{Direct grouping}: En �ste caso, el productor de la tupla de
			    datos decide a que tarea del bolt consumidor desea enviar la tupla.
			    
			    \item \emph{Local grouping}: Si el bolt de destino tiene una o m�s tareas
			    en el mismo proceso \emph{worker}, las tuplas ir�n aleatoriamente hacia
			    alquellas tareas que est�n siendo ejecutadas. En caso contrario, se
			    comportar� como un \emph{Shuffle grouping}.
			\end{itemize}
	
		\subsubsection{Tasks}
		
			Cada spout o bolt ejecuta sus tareas en el cluster de Storm. Cada tarea
			corresponde con un hilo de ejecuci�n (\emph{thread}) y los \emph{Stream
			groupings} definen como las tuplas viajan entre las tareas.
			
		\subsubsection{Workers}
		
			Las topolog�as corren sobre uno o m�s procesos \emph{worker}. Cada uno de
			estos procesos es una JVM f�sica que ejecuta un subconjunto de las tareas de
			la topolog�a.
		
	\subsection{Material}
	
		La informaci�n de �ste cap�tulo ha sido extra�da mayormente desde la
		documentaci�n de Apache Storm 1.0 \cite{ApacheStorm101}.
		
\section{Apache Spark}
\label{section_apache_spark}

	Spark es una herramienta de c�digo abierto desarrollada para procesar datos de
	manera r�pida y f�cil. Su desarrollo comenz� en 2009 en el AMPLab de la
	Universidad de Berkeley, siendo liberado su c�digo en 2010 como un proyecto
	Apache.
	
	Spark provee herramientas para procesar diversos conjuntos de datos de distinta
	naturaleza (textos, grafos, etc�tera) y datos de distintas fuentes,
	procesamiento de datos por lotes o procesamiento de un flujo de datos en tiempo
	real.
	
	Las aplicaciones para Spark pueden ser escritas en Java, Scala o Python y el
	paquete incluye mas de 80 operadores de alto nivel para trabajar con los datos.
	Adem�s de operaciones \emph{Map and Reduce} sobre Hadoop, soporta consultas SQL,
	flujos de datos y \emph{Machine Learning} \cite{Penchikala2015SparkIntro}.
	
\section{Apache Flink}
\label{section_apache_flink}

	\subsection{Conceptos\cite{ApacheFlink10Docs}}
	
		Los \emph{programas Flink} son programas comunes que implementan
		transformaciones en colecciones distribuidas, por ejemplo, filtrado,
		correspondencia, actualizaci�n de estado, uniones, agrupamientos, agregaciones,
		etc�tera. �stas colecciones se forman a partir de las fuentes de datos
		(\emph{sources}). Dichas fuentes se forman leyendo archivos, conectando Flink a
		un servidor de mensajes como Apache Kafka \ref{chapter_apache_kafka} o mediante
		colecciones definidas localmente.
		
		Los resultados de la ejecuci�n de un programa Flink son devueltos mediante el
		uso de receptores de datos \emph{sinks}. �stos receptores pueden consistir en
		escritura de archivos, impresi�n en la consola de ejecuci�n, etc�tera.
		
		Los programas Flink pueden correr localmente (standalone), embebidos en otros
		programas o en clusters.
		
		Dependiendo del tipo de fuente de datos (\emph{source}), es decir, acotados o
		no acotados, el programa Flink deber� realizar una ejecuci�n por lotes
		(\emph{batch}) o una ejecuci�n en tiempo real sobre el flujo de datos
		(\emph{straming}). Para el primer caso, se deber� utilizar la
		\textbf{\emph{DataSet API}} y para el segundo caso la \textbf{\emph{DataStream
		API}}.
		
		Los bloques b�sicos de un programa Flink son los flujos de datos (streams) y
		las transformaciones (operaciones).
	
		Al ejecutarse, un programa Flink se corresponde con lo que se conoce como
		\emph{Streaming Dataflow}. Cada \emph{Dataflow}, comienza con una o m�s fuentes
		de datos (\emph{sources}) y termina en uno o m�s receptores de datos
		(\emph{sinks}).
		En la mayor�a de los casos, existe una correspondencia uno a uno entre las
		transformaciones especificadas en el programa y las operaciones del
		\emph{Dataflow} pero puede ocurrir que una transformaci�n est� formada por mas
		de un operador de transformaci�n.
		
		\begin{figure}[H]
			\centering
			\includegraphics[width=1\linewidth]{./introduccion/img/flink/building_blocks}
			\caption{Bloques Fundamentales de un Programa Flink\cite{ApacheFlink10Docs}}
		\end{figure}
	
	\subsubsection{Usar Flink}
		
			Para escribir un programa Flink, se deben incluir las libre�as Flink en el
			proyecto, en el caso de Maven, �sto se logra insertando las siguientes l�neas
			en el pom.xml del proyecto.
		
\lstset{language=XML}
\begin{lstlisting}
<dependency>
	<groupId>org.apache.flink</groupId>
	<artifactId>flink-core_2.11</artifactId>
	<version>1.0.3</version>
</dependency>
<dependency>
	<groupId>org.apache.flink</groupId>
	<artifactId>flink-java_2.11</artifactId>
	<version>1.0.3</version>
</dependency>
<dependency>
	<groupId>org.apache.flink</groupId>
	<artifactId>flink-clients_2.11</artifactId>
	<version>1.0.3</version>
</dependency>
<dependency>
	<groupId>org.apache.flink</groupId>
	<artifactId>flink-streaming-java_2.11</artifactId>
	<version>1.0.3</version>
</dependency>
<dependency>
	<groupId>org.apache.flink</groupId>
	<artifactId>flink-connector-kafka-0.9_2.11</artifactId>
	<version>1.0.3</version>
</dependency>
\end{lstlisting}
	
		\subsubsection{DataSet y DataStreams\cite{ApacheFlink10Docs}}		
	
			Para representar datos en un programa Flink existen dos tipos de clases
			DataSet y DataStream. Se puede considerar que son colecciones inmutables de
			datos que pueden contener duplicados. En el caso del DataSet la cantidad de
			datos es finita, mientras que en un DataStream pueden ser ilimitados.
			
			�stas colecciones son diferentes a las Java en el sentido de que son
			inmutables, una vez creadas no pueden a�adirse ni removerse elementos. Tampoco
			es posible inspeccionar los elementos contenidos dentro de la colecci�n.
			
			Como se menciono anteriormente, una colecci�n DataSet o DataStream es creada
			en el momento en que se a�ade una fuente de datos \emph{source} y nuevas
			colecciones son creadas cada vez que una operaci�n de transformaci�n es
			ejecutada.
	
		\subsubsection{Evaluaci�n Postergada}
		
			Al ejecutar un programa Flink, el m�todo \emph{main} es ejecutado pero la
			carga de datos y las transformaciones no ocurren directamente. Cada operaci�n
			es creada y a�adida a un plan de ejecuci�n del programa. Las operaciones, son
			ejecutadas cuando son exlicitamente disparada mediante el llamado del m�todo
			\emph{execute()} sobre el entorno de ejecuci�n \cite{ApacheFlink10Docs}.
	
	
	\subsection{Flink DataStreams}
	
	API Programming Guide (Source: https://ci.apache.org/projects/flink/flink-docs-release-1.0/apis/streaming/index.html)
	
	DataStream programs in Flink are regular programs that implement transformations
	on data streams (e.g., filtering, updating state, defining windows,
	aggregating). The data streams are initially created from various sources (e.g.,
	message queues, socket streams, files). Results are returned via sinks, which
	may for example write the data to files, or to standard output (for example the
	command line terminal). Flink programs run in a variety of contexts, standalone,
	or embedded in other programs. The execution can happen in a local JVM, or on
	clusters of many machines.
	
	
	\subsection{Material}
	
		La informaci�n de �ste cap�tulo ha sido extra�da mayormente desde la
		documentaci�n de Apache Flink 1.0 \cite{ApacheFlink10Docs}.
 		% Desarrollo
 			\part{Desarrollo}
 				% Dise�o de la Arquitectura del Sistema
 					%%%%%%%%%%%%%%%%%%%%%%%%%%%%%%%%%%%%%%%%%%%%%%%%%%%%%%%%%%%%%%%%%%%%%%%%%%%%%%%%%
%																				%
%	TRABAJO:	Trabajo Final													%
%				Especialidad en Ingenier�a en Sistemas de Informaci�n			%
%																				%
%		Titulo:																	%
%																				%
%		Autor:	Juli�n Nonino													%
%																				%
%	Capitulo sobre Dise�o de la Arquitectura									%	
%																				%
%	A�o: 2016																	%
%																				%
%%%%%%%%%%%%%%%%%%%%%%%%%%%%%%%%%%%%%%%%%%%%%%%%%%%%%%%%%%%%%%%%%%%%%%%%%%%%%%%%%

\chapter{Dise�o de la Arquitectura del Sistema}
\label{chapter_arquitectura}


\begin{figure}[H]
	\centering
	\includegraphics[width=1\linewidth]{./informe/desarrollo/img/ArquitecturaServidor}
	\caption{Arquitectura general del Cl�ster incluyendo Zookeeper, Kafka y Storm}
\end{figure}

\begin{figure}[H]
	\centering
	\includegraphics[width=1\linewidth]{./informe/desarrollo/img/ArquitecturaConexionClientes}
	\caption{Arquitectura general del Cl�ster conectado a los programas cliente}
\end{figure}
 				% Im�genes de Docker
 					%%%%%%%%%%%%%%%%%%%%%%%%%%%%%%%%%%%%%%%%%%%%%%%%%%%%%%%%%%%%%%%%%%%%%%%%%%%%%%%%%
%																				%
%	TRABAJO:	Trabajo Final													%
%				Especialidad en Ingenier�a en Sistemas de Informaci�n			%
%																				%
%		Titulo:																	%
%																				%
%		Autor:	Juli�n Nonino													%
%																				%
%	Capitulo sobre las im�genes de Docker Creadas								%	
%																				%
%	A�o: 2016																	%
%																				%
%%%%%%%%%%%%%%%%%%%%%%%%%%%%%%%%%%%%%%%%%%%%%%%%%%%%%%%%%%%%%%%%%%%%%%%%%%%%%%%%%

\lstset
{	basicstyle=\tiny,       		% the size of the fonts that are used for the code
	numbers=left,                   % where to put the line-numbers
	numberstyle=\tiny\color{gray},  % the style that is used for the line-numbers
	stepnumber=1,                   % the step between two line-numbers. If it's 1, each line will be numbered
	numbersep=5pt,                  % how far the line-numbers are from the code
	showspaces=false,               % show spaces adding particular underscores
	showstringspaces=false,         % underline spaces within strings
	showtabs=false,                 % show tabs within strings adding particular underscores
	frame=none,                 	% adds a frame around the code
	rulecolor=\color{white},        % if not set, the frame-color may be changed on line-breaks within not-black text (e.g. comments (green here))
	tabsize=2,                      % sets default tabsize to 2 spaces
	captionpos=b,                   % sets the caption-position to bottom
	breaklines=true,                % sets automatic line breaking
	breakatwhitespace=true,        	% sets if automatic breaks should only happen at
	keywordstyle=\color{blue},     	% keyword style
  	commentstyle=\color{dkgreen}, 	% comment style
  	stringstyle=\color{gray},      	% string literal style
  	escapeinside={\%*}{*)},         % if you want to add LaTeX within your code
  	morekeywords={*,apt-get,...},   % if you want to add more keywords to the set
  	deletekeywords={local,...}      % if you want to delete keywords from the given language
}

\chapter{Im�genes de Docker}
\label{chapter_docker_images}

El sistema desarrollado se compone de una serie de im�genes de Docker que al
ejecutarse como Docker Containers, formar�n el sistema de procesamiento de datos
en tiempo real que se quiere demostrar. Las im�genes desarrolladas son:

\begin{itemize}
    \item Nodo de Zookeeper \ref{docker-image-nodo-zookeeper}
    \item Nodo de Kafka \ref{docker-image-nodo-kafka}
\end{itemize}

\section{Nodo Apache Zookeeper}
\label{docker-image-nodo-zookeeper}

	Basado en una imagen de Ubuntu 14.04 Trusty, se desarrolla la imagen de
	Apache Zookeeper.

	\lstinputlisting[language=Bash,
					 caption={Dockerfile para un nodo de Apache Zookeeper},
					 label=code_zookeeper_dockerfile
					]{./docker_images/zookeeper/Dockerfile.}
	
	Para comenzar, se realiza una actualizaci�n del sistema operativo y luego se
	procede con la instalaci�n de Apache Zookeeper a trav�s de la
	herramienta \emph{apt-get} de Ubuntu. Luego, se limpian todos los restos que
	hayan quedado de la instalaci�n para reducir el tama�o de la imagen.
	Posteriormente, se exponen los puertos necesarios para la ejecuci�n.
	
	Con eso ya se podr�a iniciar un �nico nodo de Apache Zookeeper agregando a la
	imagen el siguiente comando.
	
\lstset{language=Bash}
\begin{lstlisting}
CMD ["/usr/share/zookeeper/bin/zkServer.sh", "start-foreground"]
\end{lstlisting}

	Como se desea correr un cl�ster de Apache Zookeeper, se deben hacer ciertas
	configuraciones para que cada nodo pueda comunicarse con los dem�s. Por �sta
	raz�n se implementa el siguiente script.
	
	\lstinputlisting[language=Bash,
					 caption={Script de inicio para un nodo de Apache Zookeeper},
					 label=code_zookeeper_start_sh
					]{./docker_images/zookeeper/start.sh}

	Lo que hace el script \ref{code_zookeeper_start_sh} es recibir como par�metros
	una identificaci�n para el nodo y una lista de todos los nodos y configura Apache
	Zookeeper con esos datos antes de iniciar el servicio. Se hace de �sta manera
	debido al alcance de �sta demostraci�n, en producci�n deber�a utilizarse alg�n
	sistema de auto descubrimiento de servicios para que los nodos se encuentren
	autom�ticamente al levantarse.

	La construcci�n de �sta imagen, se realiza situado en la carpeta
	\emph{zookeeper} del repositorio, mediante el siguiente comando.
	
\lstset{language=Bash}
\begin{lstlisting}
docker build -t jnonino/zookeeper .
\end{lstlisting}

	Luego de construir la imagen de Zookeeper, el cl�ster de tres nodos puede ser
	lanzado mediante Docker Compose utilizando el siguiente comando:

\lstset{language=Bash}
\begin{lstlisting}
docker-compose -f start_zookeeper.yml up -d
\end{lstlisting}
	
	El comando anterior	inicia tres nodos de Zookeeper basados en la imagen
	construida anteriormente, el valor \emph{-d} al final indica que se desea que
	los containers corran en segundo plano.
	
	\lstinputlisting[language=Bash,
					 caption={Script de inicio de Docker Compose para levantar un cl�ster de
					 3 nodos de Apache Zookeeper},
					 label=code_zookeeper_compose
					 ]{./docker_images/zookeeper/start_zookeeper.yml}
	
	Notar que se utiliza la misma direcci�n IP para los tres nodos, �sto se debe a
	que para la demostraci�n que se hace en �ste trabajo, todos los servicios
	correran en un mismo servidor. L�gicamente, en un sistema en producci�n,
	atendiendo datos de clientes, dichos servicios deber�an correr en instancias
	diferenciadas para garantizar la alta disponibilidad que Apache Zookeeper promete
	al utilizar m�s de un nodo.
	
	Para comprobar que los tres nodos est�n corriendo correctamente, ejecutar el
	siguiente comando:

\lstset{language=Bash}
\begin{lstlisting}
for i in {2181..2183}; do 
	echo mntr | nc <IP_DEL_HOST> $i | grep zk_followers ;
done
\end{lstlisting}	

	Se debe obtener un resultado como el siguiente mostrando que Zookeeper est�
	corriendo y que hay dos nodos adem�s del nodo l�der.

\lstset{language=Bash}
\begin{lstlisting}
zk_followers	2
\end{lstlisting}

\section{Nodo de Apache Kafka}
\label{docker-image-nodo-kafka}

	De la misma manera que los nodos de Apache Zookeeper, �sta imagen es basada en la
	imagen Ubuntu 14.04 Trusty. Y, como primera medida, se procede a la actualizaci�n
	del sistema operativo.

	\lstinputlisting[language=Bash,
					 caption={Dockerfile para un nodo de Apache Kafka},
					 label=code_kafka_dockerfile
					]{./docker_images/kafka/Dockerfile.}
	
	En �ste caso, se proceden a instalar herramientas como \emph{wget}, \emph{tar} y
	\emph{Java 7 JRE} necesarias para que corra Apache Kafka. Debido a que Kafka no
	puede ser instalado mediante \emph{apt-get}, es necesario descargar el archivo
	\emph{.tar}, descomprimirlo y ejecutarlo.
	
	Luego de configurar el nodo, es posible lanzar Kafka mediante un script utilizado
	para configurar variables antes de la ejecuci�n.
	
	\lstinputlisting[language=Bash,
					 caption={Script de inicio para un nodo de Apache Kafka},
					 label=code_kafka_start_sh
					]{./docker_images/kafka/start.sh}

	La construcci�n de �sta imagen, se realiza situado en la carpeta \emph{kafka} del
	repositorio, mediante el siguiente comando.
	
\lstset{language=Bash}
\begin{lstlisting}
docker build -t jnonino/kafka .
\end{lstlisting}

	Luego de construir la imagen, el cl�ster de tres nodos puede ser lanzado mediante
	Docker Compose utilizando el siguiente comando: 

\lstset{language=Bash}
\begin{lstlisting}
docker-compose -f start_kafka.yml up -d
\end{lstlisting}
	
	El comando anterior	inicia tres nodos de Kafka basados en la imagen
	construida anteriormente, el valor \emph{-d} al final indica que se desea que
	los containers corran en segundo plano.
	
	\lstinputlisting[language=Bash,
					 caption={Script de inicio de Docker Compose para levantar un cl�ster de
					 3 nodos de Apache Kafka},
					 label=code_zookeeper_compose
					 ]{./docker_images/kafka/start_kafka.yml}
	
	Notar que se utiliza la misma direcci�n IP para los tres nodos, �sto se debe a
	que para la demostraci�n que se hace en �ste trabajo, todos los servicios
	correran en un mismo servidor. L�gicamente, en un sistema en producci�n,
	atendiendo datos de clientes, dichos servicios deber�an correr en instancias
	diferenciadas para garantizar la alta disponibilidad que Apache Kafka promete
	al utilizar m�s de un nodo.
	
	Para comprobar que los tres nodos est�n corriendo correctamente, es necesario
	descargar Kafka para interactuar con el servidor corriendo\footnote{Descargar
	Apache Kafka desde https://www.apache.org Version 0.9.0.1 con Scala 2.11}.

	Correr los siguientes comandos para crear un t�pico de prueba y obtener
	informaci�n del mismo:
	
\lstset{language=Bash}
\begin{lstlisting}
./bin/kafka-topics.sh --create --topic test --partitions 3 --zookeeper
192.168.0.104 --replication-factor 2
./bin/kafka-topics.sh --describe --topic test --zookeeper 192.168.0.104
\end{lstlisting}	

	\begin{figure}[H]
		\centering
		\includegraphics[width=1\linewidth]{./informe/desarrollo/img/kafka_creacion_topic}
	\end{figure}

	Para enviar datos al servidor de Kafka, se debe generar una lista de los nodos de
	Kafka (\emph{brokers}) para posteriormente enviarle mensajes y finalmente, con
	otra aplicaci�n provista por Kafka, leerlos.

\lstset{language=Bash}
\begin{lstlisting}
BROKER_LIST=192.168.0.104:9092,192.168.0.104:9093,192.168.0.104:9094
/bin/kafka-console-producer.sh --topic test --broker-list="$BROKER_LIST"
./bin/kafka-console-consumer.sh --zookeeper 192.168.0.104 --topic test
--from-beginning
\end{lstlisting}

	\begin{figure}[H]
		\centering
		\includegraphics[width=1\linewidth]{./informe/desarrollo/img/kafka_w_r_topic}
	\end{figure}

\section{Apache Storm}
\label{docker-images-storm}

	Tambi�n basada en una imagen de docker de Ubuntu 14.04 Trusty se crea la imagen
	base para todos los servicios de Storm. 
	Cada nodo de Storm debe correr dos aplicaciones \emph{storm-supervisor} y
	\emph{storm-logviewer}. Para correr ambos procesos en un container de Docker, se
	utiliza un sistema de control de procesos hecho en Python llamado
	\emph{supervisord}
	
	Es necesario correr tres servicios, \emph{Storm Nimbus}, \emph{Storm
	UI} y al menos un nodo \emph{Storm Supervisor}. Dado que los tres compartes
	muchas configuraciones, se crea una imagen base de Apache Storm de la cual
	derivar�n los tres servicios necesarios.
	
	\lstinputlisting[language=Bash,
					 caption={Dockerfile para la imagen base de Storm},
					 label=code_storm_base_dockerfile
					]{./docker_images/storm/storm-base/Dockerfile.}

	Se comienza actualizando el sistema operativo, luego se instalan todas las
	herramientas necesarias como \emph{tar}, \emph{wget} y \emph{supervisord}. Como
	no es posible la instalaci�n de Apache Storm desde la herramienta \emph{apt-get},
	se debe realizar un procedimiento similar al seguido con Apache Kafka.
	Por �ltimo, se copian varios archivos de configuraci�n que son necesarios para la
	ejecuci�n de supervisord y Apache Storm.

	\lstinputlisting[language=Bash,
					 caption={Archivo de configuraci�n de Apache Storm storm.yaml},
					 label=code_storm_yaml
					]{./docker_images/storm/storm-base/storm.yaml}

	\lstinputlisting[language=Bash,
					 caption={Archivo de configuraci�n de Apache Storm cluster.xml},
					 label=code_cluster_xml
					]{./docker_images/storm/storm-base/cluster.xml}
	
	\lstinputlisting[language=Bash,
					 caption={Archivo de configuraci�n de SupervisorD config-supervisord.sh},
					 label=code_config_supervisord_sh
					]{./docker_images/storm/storm-base/config-supervisord.sh}
	
	\lstinputlisting[language=Bash,
					 caption={Script de inicio de Storm},
					 label=code_start_supervisor_sh
					]{./docker_images/storm/storm-base/start-supervisor.sh}
	
	\subsection{Storm Nimbus}
	
		\lstinputlisting[language=Bash,
					 caption={Dockerfile para la imagen de Storm Nimbus},
					 label=code_storm_nimbus_dockerfile
					]{./docker_images/storm/storm-nimbus/Dockerfile.}
	
	\subsection{Storm Supervisor}
	
		\lstinputlisting[language=Bash,
					 caption={Dockerfile para la imagen de Storm Supervisor},
					 label=code_storm_supervisor_dockerfile
					]{./docker_images/storm/storm-supervisor/Dockerfile.}
					
	\subsection{Storm UI}
	
		\lstinputlisting[language=Bash,
					 caption={Dockerfile para la imagen de Storm UI},
					 label=code_storm_ui_dockerfile
					]{./docker_images/storm/storm-ui/Dockerfile.}				
	
	\subsection{Iniciar Apache Storm}
	
			Finaliza la construcci�n de todas las im�genes anteriores, es posible
			poner a correr un servidor de Apache Storm utilizando el siguiente comando:

\lstset{language=Bash}
\begin{lstlisting}
docker-compose -f start_storm.yml up -d
\end{lstlisting}
	
	El comando anterior	inicia los tres servicios de Storm necesarios, Storm
	Nimbus, Storm Supervisor y Storm UI, el valor \emph{-d} al final indica que se
	desea que los containers corran en segundo plano.
	
	\lstinputlisting[language=Bash,
					 caption={Script de inicio de Docker Compose para levantar Apache Storm},
					 label=code_storm_compose
					 ]{./docker_images/storm/start_storm.yml}
	
	Luego, es posible comprobar el estado de Storm accediendo a la interfaz web
	"http://<DOCKER_HOST_IP>:8080".
	
 				% Aplicaciones Desarrolladas
 					%%%%%%%%%%%%%%%%%%%%%%%%%%%%%%%%%%%%%%%%%%%%%%%%%%%%%%%%%%%%%%%%%%%%%%%%%%%%%%%%%
%																				%
%	TRABAJO:	Trabajo Final													%
%				Especialidad en Ingenier�a en Sistemas de Informaci�n			%
%																				%
%		Titulo:																	%
%																				%
%		Autor:	Juli�n Nonino													%
%																				%
%	Capitulo sobre el software desarrollado										%	
%																				%
%	A�o: 2016																	%
%																				%
%%%%%%%%%%%%%%%%%%%%%%%%%%%%%%%%%%%%%%%%%%%%%%%%%%%%%%%%%%%%%%%%%%%%%%%%%%%%%%%%%

\chapter{Aplicaciones Desarrolladas}
\label{chapter_aplicaciones_desarrolladas}

En �ste cap�tulo se mostrar�n las aplicaciones desarrolladas para hacer uso del
sistema de procesamiento de datos presentado en secciones
anteriores (\ref{arquitectura_general}).

Los datos ingresan al sistema a trav�s de Apache Kafka, por ello, se comenz�
generando una aplicaci�n para enviar datos al servicio de Apache Kafka y otra
para consumirlos directmente desde Apache Kafka.

Tambi�n, se desarrollo una aplicaci�n de Apache Storm, \textbf{topolog�a}, que
recuperar� los datos del servicio de Apache Kafka teniendolos disponibles para
procesarlos en Apache Storm.

	\section{Productor de Datos de Kafka}
	\label{productor_kafka}

	Para generar una entrada de datos al sistema, se implement� una aplicaci�n Java
	encargada de publicar datos en Apacker Kafka. Paralelamente, se implement� una
	aplicaci�n consumidora de datos con el objetivo de comprobar que la producci�n
	de datos es correcta.

	\begin{figure}[H]
		\centering
		\includegraphics[width=.6\linewidth]{./informe/desarrollo/img/KafkaClassDiagram}
	\end{figure}

	Existe una clase \emph{Main} encargada de recibir los par�metros de entrada de
	la aplicaci�n e inicializar hilos de ejecuci�n productores y/o consumidores de
	datos seg�n sea necesario.

	\lstinputlisting[language=Java,
					 caption={Clase Main de la aplicaci�n de Apache Kafka},
					 label=kafka_app_main
					]{./software/kafka/src/main/java/ar/edu/utn/frc/posgrado/jnonino/kafka/Main.java}
	
	Como par�metros, se reciben la lista de brokers de Apache Kafka,
	indicadores \emph{yes|no} para habilitar el hilo productor y/o el hilo
	consumidor y finalmente, un par�metro n�merico indicando, en milisegundos, la
	tasa de producci�n de mensajes.

	\newpage

	La producci�n de datos se hacer en una clase llamada
	DataProducer (\ref{kafka_app_producer}) que extiende de la clase \emph{Thread}
	de Java. Al momento de instanciar un objeto de �sta clase, se debe enviar la
	lista de brokers de Kafka y el valor de milisegundos que indica la tasa de
	producci�n de mensajes.

	\lstinputlisting[language=Java,
					 caption={Productor de datos de la aplicaci�n de Apache Kafka},
					 label=kafka_app_producer
					]{./software/kafka/src/main/java/ar/edu/utn/frc/posgrado/jnonino/kafka/DataProducer.java}

	El m�todo \emph{run} lee datos desde un archivo texto donde cada l�nea es un
	mensaje independiente y los publica en Apache Kafka. El formato de los
	mensajes incluye pais, provincia, ciudad, tiempo, temperatura, humedad,
	presi�n, velocidad y direcci�n del viento.
	
\lstset{language=Bash}
\begin{lstlisting}
Argentina,Catamarca,Aconquija,1207105200000,25.2,49.7,962,5.1,SUR
\end{lstlisting}

	\newpage
	
	Para el caso del consumidor de datos, se repite el mismo esquema de instanciar
	un objeto que extiende la clase \emph{Thread} de Java recibiendo como par�metro
	la lista de brokers de Kafka.
	
	En el m�todo \emph{run}, se recuperan mensajes desde el servidor y se los
	imprimi en la consola de usuario.

	\lstinputlisting[language=Java,
					 caption={Consumidor de prueba de la aplicaci�n de Apache Kafka},
					 label=kafka_app_consumer
					]{./software/kafka/src/main/java/ar/edu/utn/frc/posgrado/jnonino/kafka/DataConsumer.java}

	\section{Topolog�a de Procesamiento de Datos de Storm}
	\label{topologia_storm}
	
	\begin{figure}[H]
		\centering
		\includegraphics[width=1\linewidth]{./informe/desarrollo/img/StormClassDiagram}
	\end{figure}
	
	\lstinputlisting[language=Java,
					 caption={Clase Main de la topolog�a para Apache Storm},
					 label=storm_topology_main
					]{./software/storm/src/main/java/ar/edu/utn/frc/posgrado/jnonino/storm/Main.java}
	
	\lstinputlisting[language=Java,
					 caption={Bolt de procesamiento de datos de la topolog�a para Apache Storm},
					 label=storm_topology_bolt
					]{./software/storm/src/main/java/ar/edu/utn/frc/posgrado/jnonino/storm/DataProcessBolt.java}
					
	\lstinputlisting[language=Java,
					 caption={Objeto utilizado para procesar las tuplas recibidas en la
					 topolog�a para Apache Storm}, 
					 label=storm_topology_metric_record
					]{./software/storm/src/main/java/ar/edu/utn/frc/posgrado/jnonino/storm/MetricRecord.java}

 		% Resultados y Conclusiones			
 			\part{Resultados y Conclusiones}
 				% Resultados
 					%%%%%%%%%%%%%%%%%%%%%%%%%%%%%%%%%%%%%%%%%%%%%%%%%%%%%%%%%%%%%%%%%%%%%%%%%%%%%%%%%
%																				%
%	TRABAJO:	Trabajo Final													%
%				Especialidad en Ingenier�a en Sistemas de Informaci�n			%
%																				%
%		Titulo:																	%
%																				%
%		Autor:	Juli�n Nonino													%
%																				%
%	Trabajo Futuro																%	
%																				%
%	A�o: 2016																	%
%																				%
%%%%%%%%%%%%%%%%%%%%%%%%%%%%%%%%%%%%%%%%%%%%%%%%%%%%%%%%%%%%%%%%%%%%%%%%%%%%%%%%%

\chapter{Resultados}
\label{chapter_resultados}

En este cap�tulo, se mostrar� como desplegar el sistema y como ponerlo en
funcionamiento. Se comienza listando las im�genes de Docker que han sido
creadas, para mas informaci�n, ir al cap�tulo \ref{chapter_docker_images}. Esto
se hace mediante el comando:

\lstset{language=Bash}
\begin{lstlisting}
docker images
\end{lstlisting}

\begin{figure}[H]
	\centering
	\label{lista_imagenes_docker}
	\includegraphics[width=1\linewidth]{./informe/conclusiones/img/ImagenesDocker}
	\caption{Listado de im�genes de Docker creadas para el sistema}
\end{figure}

Mediante Docker Compose, se hace el despliegue de los cl�ster de Apache
Zookeeper, Apache Kafka y Apache Storm.

\begin{figure}[H]
	\centering
	\label{inicio_cluster_zookeeper}
	\includegraphics[width=1\linewidth]{./informe/conclusiones/img/IniciarClusterZookeeper}
	\caption{Inicio del cl�ster de Apache Zookeeper}
\end{figure}

\newpage

Luego de esperar algunos minutos, se procede a iniciar el cl�ster de Apache
Kafka. Esto se hace para garantizar que Apache Zookeeper est� listo para recibir conexiones.

\begin{figure}[H]
	\centering
	\label{inicio_cluster_kafka}
	\includegraphics[width=1\linewidth]{./informe/conclusiones/img/IniciarClusterKafka}
	\caption{Inicio del cl�ster de Apache Kafka}
\end{figure}

Esperando algunos minutos para asegurar que Apache Kafka est� corriendo
correctamente, se despliegue el cl�ster de Apache Storm.

\begin{figure}[H]
	\centering
	\label{inicio_cluster_storm}
	\includegraphics[width=1\linewidth]{./informe/conclusiones/img/IniciarClusterStorm}
	\caption{Inicio del cl�ster de Apache Storm}
\end{figure}

Si no se tratara de una prueba de concepto y fuera un sistema productivo real,
se deber�a establecer un mecanismo de sincronizaci�n autom�tico para garantizar
que un servicio est� disponible antes de iniciar el siguiente. A los fines de
esta prueba de concepto, es suficiente con esperar algunos minutos antes de
desplegar cada servicio.

Cuando todo el sistema se encuentra corriendo correctamente, es posible acceder
a la interfaz gr�fica de Apache Storm utilizando la direcci�n
\url{http://<IP_SERVICIO_STORM_UI>:8080}.

\newpage

\begin{figure}[H]
	\centering
	\label{storm_ui_limpio}
	\includegraphics[width=.95\linewidth]{./informe/conclusiones/img/UIStorm}
	\caption{Interfaz gr�fica de Apache Storm con el sistema limpio}
\end{figure}

\section{Compilaci�n de aplicaciones}

	Ambas aplicaciones desarrolladas, tanto el productor de datos de Apache Kafka
	como la topolog�a de Apache Storm, han sido creadas como un multi-proyecto de
	Apache Maven. De esta manera, se pueden compilar al mismo tiempo con el comando
	\emph{mvn clean install}.

	\begin{figure}[H]
		\centering
		\label{storm_ui_limpio}
		\includegraphics[width=1\linewidth]{./informe/conclusiones/img/CompilandoAplicaciones}
		\caption{Compilaci�n de las aplicaciones Java desarrolladas}
	\end{figure}

\section{Despliegue de la topolog�a de Apache Storm}

	Para el despliegue de la topolog�a de Apache Storm, se debe contar con el
	ejecutable descargado desde el sitio
	oficial\footnote{http://storm.apache.org/downloads.html} version 0.9.7, esto
	permite tener el comando \emph{storm} disponible en la l�nea de comandos.

\lstset{language=Bash}
\begin{lstlisting}
storm jar storm-1.0-SNAPSHOT-jar-with-dependencies.jar
ar.edu.utn.frc.posgrado.jnonino.storm.Main <IP_NODO_ZOOKEEPER>:<PUERTO_NODO_ZOOKEEPER>
\end{lstlisting}

	\begin{figure}[H]
		\centering
		\label{comando_despliegue_topologia}
		\includegraphics[width=.8\linewidth]{./informe/conclusiones/img/SubiendoTopologiaStorm}
		\caption{Ejecuci�n del comando de despliegue de la topolog�a de Apache Storm}
	\end{figure}

	La siguiente imagen muestra en la interfaz gr�fica de Apache Storm que la
	topolog�a fue desplegada correctamente.

	\begin{figure}[H]
		\centering
		\label{topologia_desplegada}
		\includegraphics[width=.6\linewidth]{./informe/conclusiones/img/TopologiaSubida}
		\caption{Topolog�a de Apache Storm desplegada correctamente}
	\end{figure}

	Luego de dicho procedimiento, la topolog�a queda lista para recibir datos
	nuevos y procesarlos.

	\newpage
	
\section{Producci�n de datos}

	La producci�n de datos se realiza con una aplicaci�n Java desarrollada para
	conectarse con Apache Kafka y enviarle mensajes. El comando para la ejecuci�n
	de dicha aplicaci�n comienza con \emph{java -jar <JAR-A-UTILIZAR>} seguido de
	los argumentos de ejecuci�n. El primer argumento es lista de \emph{brokers} de
	Apache Kafka. Le sigue un par�metro \emph{yes|no} habilitando o deshabilitando
	la producci�n de mensajes y un argumento de numero entero indicando en
	milisegundos la tasa de producci�n de mensajes. Y, finalmente, otro par�metro
	\emph{yes|no} indicando si se habilita o no el consumidor de datos de prueba.

	\begin{figure}[H]
		\centering
		\label{inicio_produccion_datos}
		\includegraphics[width=.82\linewidth]{./informe/conclusiones/img/KafkaProduciendoDatos1}
		\caption{Comienzo de la producci�n de datos de Apache Kafka}
	\end{figure}

	En la siguiente imagen (\ref{produccion_datos}) se observar� como la aplicaci�n
	registra en la consola cada dato que va enviando al cl�ster de Apache Kafka.

	\begin{figure}[H]
		\centering
		\label{produccion_datos}
		\includegraphics[width=.82\linewidth]{./informe/conclusiones/img/KafkaProduciendoDatos2}
		\caption{Producci�n de datos de Apache Kafka}
	\end{figure}

	\newpage

\section{Procesamiento de datos}

	Con la aplicaci�n productora de datos enviando mensajes a Apache Kafka, la
	topolog�a de Apache Storm comenzara a recolectarlos a trav�s de un
	\emph{spout}, ver imagen \ref{spout_storm}.

	\begin{figure}[H]
		\centering
		\label{spout_storm}
		\includegraphics[width=.92\linewidth]{./informe/conclusiones/img/StormProcesandoDatosSpout}
		\caption{Estad�sticas de procesamiento del \emph{spout} de Apache Storm}
	\end{figure}
	
	Luego, el \emph{bolt}, es encargado de recibir dichos mensajes desde el
	\emph{spout} y procesarlos, para esta prueba de concepto, el procesamiento
	consiste en imprimir un registro del mensaje procesado en los registros de
	Apache Storm. Aplicaciones productivas reales, podr�an hacer agregaci�n de
	datos, lanzar alarmas ante alg�n mensaje en particular, etc�tera. Ver imagen
	\ref{bolt_storm}.
	
	\begin{figure}[H]
		\centering
		\label{bolt_storm}
		\includegraphics[width=.92\linewidth]{./informe/conclusiones/img/StormProcesandoDatosBolt}
		\caption{Estad�sticas de procesamiento del \emph{bolt} de Apache Storm}
	\end{figure}
	
	\newpage
	
	La siguiente imagen (figura \ref{logs_storm}) muestra desde la interfaz gr�fica
	de Apache Storm los registros de ejecuci�n (\emph{logs}). En ellos, se puede
	apreciar como el \emph{bolt} de procesamiento est� haciendo los registros
	correspondientes tal como fue programado, ver secci�n \ref{topologia_storm}.
	
	\begin{figure}[H]
		\centering
		\label{logs_storm}
		\includegraphics[width=1\linewidth]{./informe/conclusiones/img/StormProcesandoDatos}
		\caption{Registros de Apache Storm mostrando los mensajes que est�n siendo procesados}
	\end{figure}
	
	\newpage
	
\section{Limpieza del sistema}

	Para detener el sistema, se utiliza el comando \emph{docker stop
	<numero-contenedor>}. De esta manera, se procede a detener cada uno de los
	contenedores corriendo.
	
	\begin{figure}[H]
		\centering
		\label{deteniendo_contenedores}
		\includegraphics[width=1\linewidth]{./informe/conclusiones/img/DeteniendoContenedores}
		\caption{Deteniendo contenedores de Docker}
	\end{figure}
	
	Luego, si se desea, es posible remover esos contenedores detenidos mediante el
	comando \emph{docker rm <numero-contenedor>}, dejando el sistema limpio
	nuevamente.
	
	\begin{figure}[H]
		\centering
		\label{removiendo_contenedores}
		\includegraphics[width=1\linewidth]{./informe/conclusiones/img/RemoviendoContenedores}
		\caption{Removiendo contenedores de Docker}
	\end{figure}
	
 				% Conclusiones
 					%%%%%%%%%%%%%%%%%%%%%%%%%%%%%%%%%%%%%%%%%%%%%%%%%%%%%%%%%%%%%%%%%%%%%%%%%%%%%%%%%
%																				%
%	TRABAJO:	Trabajo Final													%
%				Especialidad en Ingenier�a en Sistemas de Informaci�n			%
%																				%
%		Titulo:	Procesamiento de Datos en Tiempo Real							%
%																				%
%		Autor:	Juli�n Nonino													%
%																				%
%	Conclusiones y Trabajo Futuro												%
%																				%
%	A�o: 2016																	%
%																				%
%%%%%%%%%%%%%%%%%%%%%%%%%%%%%%%%%%%%%%%%%%%%%%%%%%%%%%%%%%%%%%%%%%%%%%%%%%%%%%%%%

\chapter{Conclusiones y Trabajo Futuro}

\section{Conclusiones}
\label{conclusiones}
	
	El procesamiento de flujos de datos es requerido y toma mucha relevancia cuando
	cada dato debe ser procesado r�pidamente y/o continuamente, por ejemplo, cuando
	hay que tomar acciones en tiempo real\cite{Wahner2014}. Los principales
	impulsores de �stas tecnolog�as son Big Data y el Internet de las Cosas
	(\emph{Internet of Things}).


\section{Trabajo Futuro}
\label{trabajo_futuro}

	Este trabajo puede ser continuado y mejorado desde varias aristas, ampliando la
	prueba de concepto hasta llegar a implementar un sistema de procesamiento de
	datos en tiempo real, desplegado en la nube, con capacidad de crecer y decrecer
	de acuerdo a la carga del sistema.
	
	\begin{itemize}
	    \item Dise�ar el implementar \emph{bolts} de Apache Storm para tomar
	    \item Utilizar un proveedor de servicios en la nube como Amazon Web Services,
	    Microsoft Azure, Google Cloud Platform, etc�tera.
	    \item Plantear, implementar y probar mecanismos de auto escalabilidad
	    (crecimiento y decrecimiento) del sistema de acuerdo con la carga de datos.
	    \item Plantear, implementar y probar un subsistema de almacenamiento de datos
	    utilizando tecnolog�as bases de datos como Apache HBase, Cassandra, etc�tera.
	    \item Plantear, implementar y probar un subsistema de an�lisis de datos y
	    generaci�n de reportes (\emph{analytics}).
	    \item Dise�ar y plantear modificaciones en el sistema de acuerdo a un modelo
	    de micro servicios permitiendo activar y desactivar caracter�sticas del
	    sistema f�cilmente. Adem�s, nuevas caracter�sticas podr�an ser agregadas sin
	    afectar el resto de la funcionalidad del sistema.
	\end{itemize}

 	
	% APENDICES
		%\appendix
		%\part{Ap�ndices}
		
			% Nuevo proyecto en Xilinx XPS
				%\input{./informe/apendices/nuevos_proyectos_XPS/nuevos_proyectos_XPS}
			% Creacion de un nuevo IP core	
				%\input{./informe/apendices/creacion_IP_core/creacion_IP_core}
			% Interrupciones en los IP core
				%\input{./informe/apendices/interrupciones/interrupciones}
			% Gesti�n de la metodolog�a de desarrollo
				%%%%%%%%%%%%%%%%%%%%%%%%%%%%%%%%%%%%%%
%									%
%	Copyright 2014 - Julian Nonino	%								
%									%
%%%%%%%%%%%%%%%%%%%%%%%%%%%%%%%%%%%%%

\section{Metodolog�a de desarrollo}

	El supuesto de �sta tesis consiste en que es posible dise�ar e implementar un
	herramienta de ense�anza en l�nea que permita a los estudiantes relacionar
	t�cnicas de estimaci�n y planeamiento utilizadas en Scrum con pr�cticas
	tradicionales de estimaci�n utilizando planeamiento de escenarios y generaci�n
	autom�tica de datos para pron�stico.

	Basado en el supuesto anterior, la primera etapa de la tesis consistir� en la
	recolecci�n y an�lisis de material sobre el tema. Se buscar�n libros, art�culos
	y presentaciones de congresos y conferencias, \emph{white papers}, art�culos de
	revistas especializadas, adem�s de blog, sitios de noticias, foros, de opini�n,
	etc�tera.

	Los puntos m�s importantes para la realizaci�n de la Revisi�n Sistem�tica de la
	Literatura son:
	\begin{itemize}
		\item Metodolog�as �giles (Agile). 
		\item Scrum.
		\item Pr�cticas de estimaci�n y planeamiento en Scrum.
		\item Herramientas existentes para soporte de los procesos de estimaci�n y
		planeamiento en Scrum.
		\item Historias de usuario, puntos de historia, horas ideales, velocidad.
		\item Proyectos de alcance fijo, duraci�n fija o precio fijo.
		\item Estimaci�n y planeamiento para m�ltiples equipos en Scrum.
		\item Simulaci�n de Monte Carlo.
		\item Simulaci�n de escenarios alternativos.
		\item Ense�anza de las pr�cticas de estimaci�n y planeamiento utilizadas en
		Scrum a alumnos de novel universitario.
	\end{itemize}

	Analizada la bibliograf�a, se deber�n determinar los requerimientos de la
	herramienta y se deber� elaborar un dise�o que contemple las pr�cticas de
	estimaci�n y planeamiento utilizadas en Scrum, combinadas con pr�cticas
	avanzadas pensando en el �mbito universitario, profesores y alumnos, como
	destinatario de la herramienta.

	Luego, se comenzar� la construcci�n del primer prototipo de la herramienta en
	Microsoft Excel pensando la simplicidad de uso de la misma y en la claridad de
	los conceptos utilizados para que sea �til en el �mbito universitario. Dicha
	herramienta podr�a ser probada por alumnos durante los cursos de Ingenier�a de
	Software y Gesti�n de la Calidad del Software pertenecientes a la carrera
	Ingenier�a en Computaci�n de la Universidad Nacional de C�rdoba. Se deber�n
	definir m�tricas para la evaluaci�n del impacto que produzca la herramienta
	sobre los alumnos e identificar falencias, errores y mejoras para continuar el
	desarrollo.
	
	Los resultados obtenidos deber�n ser analizados detenidamente con el fin de
	identificar posibles correcciones y mejoras en los requerimientos y dise�os
	originales, considerando que la herramienta deber� ser desarrollada como
	aplicaci�n web disponible en l�nea.

	Corregidos y mejorados los requerimientos originales e identificados nuevos
	requerimientos y conceptos que deben ser implementados en la segunda versi�n de
	la herramienta, se comenzar� con la construcci�n de la misma.
	
	Para comenzar con �sta etapa se deben analizar las distintas alternativas en
	lenguajes de programaci�n, frameworks y otras herramientas utilizadas en el
	mercado para el desarrollo de aplicaciones web. Adem�s, se deben analizar
	alternativas para hacer que la aplicaci�n web desarrollada pueda ser puesta en
	l�nea.
	
	Estando construida la herramienta, puede ser probada nuevamente en el �mbito
	universitario, como se mencion� anteriormente. Dicha prueba generar� resultados
	y estad�sticas de uso, ser�n encontrados errores y mejoras y, ser�n
	identificados fortalezas y beneficios generados por la existencia de la
	herramienta. Los datos recolectados ser�n analizados, contrastados con la
	hip�tesis y objetivos planteados al comienzo del desarrollo y luego plasmados
	en el informe de tesis con las respectivas conclusiones.
	
	\newpage

			% C�digos en Verilog de los IP cores
				%\input{./informe/apendices/codigos_ipcores/codigos_ipcores}
			% C�digos en C de los programas ejecutados
				%\input{./informe/apendices/codigos_programas/codigos_programas}
	
	%INICIO DE LA PARTE FINAL. Bibliografia e �ndices
		\backmatter
		% Referencias
			\addcontentsline{toc}{chapter}{Bibliograf�a}
			\printbibliography
		% Indices
			% Indice de contenido
				\addcontentsline{toc}{chapter}{�ndice de Contenido}
				\tableofcontents
			% �ndice de Figuras
				\cleardoublepage
				\addcontentsline{toc}{chapter}{�ndice de Figuras}
				\listoffigures

\end{document}
