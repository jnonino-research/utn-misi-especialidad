%%%%%%%%%%%%%%%%%%%%%%%%%%%%%%%%%%%%%%%%%%%%%%%%%%%%%%%%%%%%%%%%%%%%%%%%%%%%%%%%%
%																				%
%	TRABAJO:	Trabajo Final													%
%				Especialidad en Ingenier�a en Sistemas de Informaci�n			%
%																				%
%		Titulo:																	%
%																				%
%		Autor:	Juli�n Nonino													%
%																				%
%	Capitulo sobre las im�genes de Docker Creadas								%	
%																				%
%	A�o: 2016																	%
%																				%
%%%%%%%%%%%%%%%%%%%%%%%%%%%%%%%%%%%%%%%%%%%%%%%%%%%%%%%%%%%%%%%%%%%%%%%%%%%%%%%%%

\lstset
{	basicstyle=\tiny,       		% the size of the fonts that are used for the code
	numbers=left,                   % where to put the line-numbers
	numberstyle=\tiny\color{gray},  % the style that is used for the line-numbers
	stepnumber=1,                   % the step between two line-numbers. If it's 1, each line will be numbered
	numbersep=5pt,                  % how far the line-numbers are from the code
	showspaces=false,               % show spaces adding particular underscores
	showstringspaces=false,         % underline spaces within strings
	showtabs=false,                 % show tabs within strings adding particular underscores
	frame=none,                 	% adds a frame around the code
	rulecolor=\color{white},        % if not set, the frame-color may be changed on line-breaks within not-black text (e.g. comments (green here))
	tabsize=2,                      % sets default tabsize to 2 spaces
	captionpos=b,                   % sets the caption-position to bottom
	breaklines=true,                % sets automatic line breaking
	breakatwhitespace=true,        	% sets if automatic breaks should only happen at
	keywordstyle=\color{blue},     	% keyword style
  	commentstyle=\color{dkgreen}, 	% comment style
  	stringstyle=\color{gray},      	% string literal style
  	escapeinside={\%*}{*)},         % if you want to add LaTeX within your code
  	morekeywords={*,apt-get,...},   % if you want to add more keywords to the set
  	deletekeywords={local,...}      % if you want to delete keywords from the given language
}

\chapter{im�genes de Docker}
\label{chapter_docker_images}

El sistema desarrollado se compone de una serie de im�genes de Docker que al
ejecutarse como Docker Containers, formar�n el sistema de procesamiento de datos
en tiempo real que se quiere demostrar. Las im�genes desarrolladas son:

\begin{itemize}
    \item Ubuntu Base \ref{docker-image-ubuntu-base}
    \item Nodo de Zookeeper \ref{docker-image-nodo-zookeeper}
    \item Nodo de Kafka \ref{docker-image-nodo-kafka}
\end{itemize}

\section{Ubuntu Base}
\label{docker-image-ubuntu-base}

	Es la base del sistema, todos los nodos de los diferentes componentes correr�n en
	un sistema operativo Ubuntu 16.04 Xenial.
	
	\lstinputlisting[language=Bash,
					 caption={Dockerfile para la imagen base Ubuntu 16.04 Xenial actualizado},
					 label=code_ubuntu_base_dockerfile
					]{./docker_images/ubuntu-base/Dockerfile.}
					
	Si bien cada una de las siguientes im�genes podr�a haber comenzado con el c�digo
	anterior, se considera apropiada la separaci�n para evitar el proceso de
	actualizaci�n del sistema operativo durante la construcci�n de cada una de las
	im�genes.
	
	La construcci�n de �sta imagen, se realiza mediante el comando situado
	previamente en la carpeta \emph{ubuntu-base} del repositorio.
	
\lstset{language=bash}
\begin{lstlisting}
docker build -t jnonino/ubuntu-base:16.04 .
\end{lstlisting}
	
\section{Cl�ster Apache Zookeeper}
\label{docker-image-nodo-zookeeper}

	

	\lstinputlisting[language=Bash,
					 caption={Dockerfile para un nodo de Apache Zookeeper},
					 label=code_zookeeper_dockerfile
					]{./docker_images/zookeeper/Dockerfile.}
	
	\lstinputlisting[language=Bash,
					 caption={Script de inicio para un nodo de Apache Zookeeper},
					 label=code_zookeeper_start_sh
					]{./docker_images/zookeeper/start.sh}

\section{Cl�ster Apache Kafka}
\label{docker-image-nodo-kafka}

	\lstinputlisting[language=Bash,
					 caption={Dockerfile para un nodo de Apache Kafka},
					 label=code_kafka_dockerfile
					]{./docker_images/kafka/Dockerfile.}
	
	\lstinputlisting[language=Bash,
					 caption={Script de inicio para un nodo de Apache Kafka},
					 label=code_kafka_start_sh
					]{./docker_images/kafka/start.sh}

\section{Cl�ster Apache Storm}
\label{docker-image}