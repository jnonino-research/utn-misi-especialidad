%%%%%%%%%%%%%%%%%%%%%%%%%%%%%%%%%%%%%
%									%
%	Copyright 2014 - Julian Nonino	%								
%									%
%%%%%%%%%%%%%%%%%%%%%%%%%%%%%%%%%%%%%

\section{Justificacion del tema elegido}

	En la actualidad, no existe una herramienta o m�todo que permita a los
	estudiantes simular diferentes escenarios de planeamiento de release utilizando
	pr�cticas avanzadas de estimaci�n\footnote{Estimaci�n de tres puntos,
	simulaci�n de Monte Carlo, etc�tera.}, sin incurrir en la necesidad
	de crear un backlog real en una herramienta espec�fica como Rational Team
	Concert\footnote{http://www-03.ibm.com/software/products/es/rtc},
	Rally\footnote{https://www.rallydev.com/} o
	VersionOne\footnote{http://www.versionone.com/}, por citar algunas de las
	herramientas del mercado.
	Lo cual consume mucho tiempo y tiene una flexibilidad muy limitada.
	
	Como resultado de �sta tesis se espera identificar y recolectar las pr�cticas
	de estimaci�n y planeamiento utilizadas en Scrum y construir una herramienta,
	disponible en l�nea, que permita ense�ar dichas pr�cticas a alumnos de nivel
	universitario. Adem�s, se espera que durante el proceso de an�lisis de
	bibliograf�a y construcci�n de la herramienta se identifiquen las mejores
	pr�cticas utilizadas en Scrum con el fin de proveer lineamientos de estimaci�n
	y planeamiento extensibles a la industria del software.
