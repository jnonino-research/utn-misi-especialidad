%%%%%%%%%%%%%%%%%%%%%%%%%%%%%%%%%%%%%
%									%
%	Copyright 2014 - Julian Nonino	%								
%									%
%%%%%%%%%%%%%%%%%%%%%%%%%%%%%%%%%%%%%

\section{Objetivos del trabajo}

	En �sta secci�n se presentar�n los objetivos principales y secundarios de la
	tesis de maestr�a a realizar.

	\subsection{Objetivos Principales}
	
	El objetivo principal de �ste trabajo es:
	
	\begin{quote}
		\emph{Construir una herramienta para la ense�anza de t�cnicas y m�todos de
		estimaci�n y planeamiento en Scrum.}
	\end{quote}
	
	\subsection{Objetivos Secundarios}
	
	Como objetivos secundarios se tiene:
	\begin{itemize}
	  	\item Realizar una Revisi�n Sistem�tica de la Literatura sobre t�cnicas y
	  	m�todos de planeamiento y estimaci�n utilizados en metodolog�as �giles, 
	  	contemplando pr�cticas avanzadas como variabilidad en la cantidad y tama�o
	  	de los equipos de desarrollo, simulaci�n de Monte Carlo, alcance y/o esfuerzo
	  	fijos y variables, etc�tera. Adem�s, analizar el material bibliogr�fico
	  	existente relacionado con la ense�anza de Scrum y de los m�todos de
	  	estimaci�n y planeamiento utilizados.
	  	\item Determinar c�mo pr�cticas avanzadas de estimaci�n y planeamiento
	  	pueden ser combinadas con los conceptos de estimaci�n en Scrum e incluirlas
	  	en la herramienta.
	  	\item Identificar y proveer lineamientos de estimaci�n y planeamiento en
	  	Scrum que sean extensibles a la industria de desarrollo de software nacional
	  	e internacional.
	  	\item Dado que la herramienta debe estar disponible en l�nea, analizar
	  	alternativas para el \emph{hosting} de la misma.
	  	\item Analizar, dise�ar e implementar mecanismos de conexi�n y recolecci�n
	  	de m�tricas desde repositorios gratuitos de c�digo como Github, Google Code,
	  	BitBucket, etc�tera. Entre las m�tricas a recolectar o derivar se pueden nombrar
	  	cantidad de l�neas de c�digo, tasas de defectos, etc�tera.
	\end{itemize}
	