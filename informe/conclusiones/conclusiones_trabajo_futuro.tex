%%%%%%%%%%%%%%%%%%%%%%%%%%%%%%%%%%%%%%%%%%%%%%%%%%%%%%%%%%%%%%%%%%%%%%%%%%%%%%%%%
%																				%
%	TRABAJO:	Trabajo Final													%
%				Especialidad en Ingenier�a en Sistemas de Informaci�n			%
%																				%
%		Titulo:	Procesamiento de Datos en Tiempo Real							%
%																				%
%		Autor:	Juli�n Nonino													%
%																				%
%	Conclusiones y Trabajo Futuro												%
%																				%
%	A�o: 2016																	%
%																				%
%%%%%%%%%%%%%%%%%%%%%%%%%%%%%%%%%%%%%%%%%%%%%%%%%%%%%%%%%%%%%%%%%%%%%%%%%%%%%%%%%

\chapter{Conclusiones y Trabajo Futuro}

\section{Conclusiones}
\label{conclusiones}

	En este trabajo se ha logrado implementar una prueba de concepto utilizando
	Docker como mecanismo de generaci�n de la infraestructura. Se ha mostrado c�mo
	utilizar una tecnolog�a como \emph{Apache Kafka} para recibir y reenviar
	mensajes para que sean procesados en el sistema. Tambi�n, el despliegue y
	configuraci�n de \emph{Apache Storm} para procesar dichos mensajes. Adem�s, se
	muestra el uso de \emph{Apache Zookeeper} como mecanismo de coordinaci�n entre
	estos servicios.

	Por otro lado, se muestra c�digo de aplicaciones desarrolladas para la
	publicaci�n y lectura de datos en Apache Kafka. De la misma manera, se
	implementa una topolog�a de Apache Storm para leer datos desde el servicio de
	Apache Kafka y procesarlos en tiempo real, a medida que son recibidos en el
	sistema.

	El procesamiento de flujos de datos es requerido y toma mucha relevancia cuando
	cada dato debe ser procesado r�pidamente y/o continuamente, por ejemplo, cuando
	hay que tomar acciones en tiempo real\cite{Wahner2014}, es especialmente
	importante en productos Big Data e Internet de las Cosas (\emph{Internet of
	Things}).
	
	A modo de conclusion, es posible decir que 

\section{Trabajo Futuro}
\label{trabajo_futuro}

	Este trabajo puede ser continuado y mejorado desde varias aristas, ampliando la
	prueba de concepto hasta llegar a implementar un sistema de procesamiento de
	datos en tiempo real, desplegado en la nube, con capacidad de crecer y decrecer
	de acuerdo a la carga del sistema.
	
	\begin{itemize}
	    \item Dise�ar e implementar \emph{bolts} de Apache Storm para realizar un
	    procesamiento de datos m�s complejo, como por ejemplo, tomar promedios,
	    calcular m�ximos y m�nimos, evaluar reglas para enviar notificaciones o
	    ejecutar acciones ante alg�n determinado valor, etc�tera.
	    \item Utilizar un proveedor de servicios en la nube como Amazon Web Services,
	    Microsoft Azure, Google Cloud Platform, etc�tera.
	    \item Plantear, implementar y probar mecanismos de auto escalabilidad
	    (crecimiento y decrecimiento) del sistema de acuerdo con la carga de datos.
	    \item Plantear, implementar y probar un subsistema de almacenamiento de datos
	    utilizando tecnolog�as bases de datos como Apache HBase, Cassandra, etc�tera.
	    \item Plantear, implementar y probar un subsistema de an�lisis de datos y
	    generaci�n de reportes (\emph{analytics}).
	    \item Dise�ar y plantear modificaciones en el sistema de acuerdo a un modelo
	    de micro servicios permitiendo activar y desactivar caracter�sticas del
	    sistema f�cilmente. Adem�s, nuevas caracter�sticas podr�an ser agregadas sin
	    afectar el resto de la funcionalidad del sistema.
	\end{itemize}
