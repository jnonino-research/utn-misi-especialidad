%%%%%%%%%%%%%%%%%%%%%%%%%%%%%%%%%%%%%%%%%%%%%%%%%%%%%%%%%%%%%%%%%%%%%%%%%%%%%%%%%
%																				%
%	TRABAJO:	Trabajo Final													%
%				Especialidad en Ingenier�a en Sistemas de Informaci�n			%
%																				%
%		Titulo:																	%
%																				%
%		Autor:	Juli�n Nonino													%
%																				%
%	Conclusiones																%	
%																				%
%	A�o: 2016																	%
%																				%
%%%%%%%%%%%%%%%%%%%%%%%%%%%%%%%%%%%%%%%%%%%%%%%%%%%%%%%%%%%%%%%%%%%%%%%%%%%%%%%%%

\chapter{Conclusiones}





https://www.infoq.com/articles/stream-processing-hadoop
Conclusion \cite{Wahner2014}

Stream processing is required when data has to be processed fast and / or
continuously, i.e. reactions have to be computed and initiated in real time.
This requirement is coming more and more into every vertical. Many different
frameworks and products are available on the market already, however the number
of mature solutions with good tools and commercial support is small today.
Apache Storm is a good, open source framework; however custom coding is required
due to a lack of development tools and there is no commercial support right now.
Products such as IBM InfoSphere Streams or TIBCO StreamBase offer complete
products, which close this gap. You definitely have to try out the different
products, as the websites do not show you how they differ regarding ease of use,
rapid development and debugging, and real-time streaming analytics and
monitoring. Stream processing complements other technologies such as a DWH and
Hadoop in a big data architecture - this is not an "either/or" question. Stream
processing has a great future and will become very important for most companies.
Big Data and Internet of Things are huge drivers of change.


\section{Trabajo Futuro}

	Este trabajo puede ser continuado y mejorado desde varias aristas, ampliando la
	prueba de concepto hasta llegar a implementar un sistema de procesamiento de
	datos en tiempo real, desplegado en la nube, con capacidad de crecer y decrecer
	de acuerdo a la carga del sistema.
	
	\begin{itemize}
	    \item Utilizar un proveedor de servicios en la nube como Amazon Web Services,
	    Microsoft Azure, Google Cloud Platform, etc�tera.
	    \item Plantear, implementar y probar mecanismos de auto escalabilidad
	    (crecimiento y decrecimiento) del sistema de acuerdo con la carga de datos.
	    \item Plantear, implementar y probar un subsistema de almacenamiento de datos
	    utilizando tecnolog�as bases de datos como Apache HBase, Cassandra, etc�tera.
	    \item Plantear, implementar y probar un subsistema de an�lisis de datos y
	    generaci�n de reportes (\emph{analytics}).
	    \item Dise�ar y plantear modificaciones en el sistema de acuerdo a un modelo
	    de micro servicios permitiendo activar y desactivar caracter�sticas del
	    sistema f�cilmente. Adem�s, nuevas caracter�sticas podr�an ser agregadas sin
	    afectar el resto de la funcionalidad del sistema.
	\end{itemize}




