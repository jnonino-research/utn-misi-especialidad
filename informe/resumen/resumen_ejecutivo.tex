%%%%%%%%%%%%%%%%%%%%%%%%%%%%%%%%%%%%%%%%%%%%%%%%%%%%%%%%%%%%%%%%%%%%%%%%%%%%%%%%%
%																				%
%	TRABAJO:	Trabajo Final													%
%				Especialidad en Ingenier�a en Sistemas de Informaci�n			%
%																				%
%		Titulo:	Procesamiento de Datos en Tiempo Real							%
%																				%
%		Autor:	Juli�n Nonino													%
%																				%
%	Resumen Ejecutivo															%	
%																				%
%	A�o: 2016																	%
%																				%
%%%%%%%%%%%%%%%%%%%%%%%%%%%%%%%%%%%%%%%%%%%%%%%%%%%%%%%%%%%%%%%%%%%%%%%%%%%%%%%%%

\chapter*{Resumen Ejecutivo}

En los �ltimos a�os, con las llegada de las Redes Sociales, Big Data, Internet
de las Cosas, entre otras, la cantidad de datos generados creci�
exponencialmente. Paralelamente, se vi� incrementada la necesidad de tomar
acciones en base a dichos datos en el menor tiempo posible. En medio de �ste
fen�meno, surgen los sistemas en la nube, con escalabilidad para crecer y
decrecer en base a los requerimientos de procesamiento de cada momento. En ese
marco surgen nuevas tecnolog�as como \emph{Docker}, \emph{Apache Zookeeper},
\emph{Apache kafka} y \emph{Apache Storm}.

Con \emph{Docker} es posible utilizar contenedores de software independientes
que pueden comunicarse con otros contenedores proporcionando una capa de
virtualizaci�n sencilla a nivel de sistema operativo Linux. Su principal
ventaja radica en la facilidad de despligue de apliacaci�n que provee su uso.

\emph{Apache Zookeeper} surge como un servicio de coordinaci�n de alto
rendimiento para aplicaciones distribuidas. Expone servicios comunes, tales como
nomenclatura, manejo de las configuraciones, sincronizaci�n, etc�tera. Entre
otras funciones, se utiliza para implementar consenso, manejar grupos, elecci�n
de nodo l�der y protocolos de presencia.

\emph{Apache Kafka} es un sistema de mensajes distrubuido, particionado y con
replicaci�n. Su principal objetivo es recibir datos desde el mundo exterior al
sistema y garantizar su diponibildiad para otros componentes del sistema que
necesiten leerlos y procesarlos.

\emph{Apache Storm} es una herramienta de procesamiento de datos en tiempo real
de c�digo abierto y gratuita creada por Twitter y luego liberada bajo la �rbita
de los proyectos Apache. La finalidad de Storm es proveer un mecanismo confiable
para procesamiento de flujos de datos ilimitados. De acuerdo a su documentaci�n,
Storm es capaz de procesar un mill�n de tuplas de datos por segundo por nodo.
Provee caracter�sticas de escalabilidad, tolerancia a fallos, garant�as de que
todos los datos ser�n procesados, etc�tera\cite{ApacheStorm097}.

Utilizando dichas tecnolog�as se generan im�genes de Docker para la conformaci�n
de un cl�ster de Apache Zookeeper, uno de Apache Kafka y otro de Apache Storm.
Se utilizar� Docker Compose para levantar los servicios necesarios. Para el
alcance de �sta prueba de concepto, todos los servicios correr�n en el mismo
servidor pero, para aprovechar las propuesta de alta disponibilidad y
escalabilidad que ofrecen �stas tecnolog�as, ser�a recomedable que cada nodo de
cada uno de los servicios corra en un servidor independiente.

Se desarrolla una aplicaci�n Java para enviar datos a Apache Kafka en el sistema
y una topolog�a de Apache Storm que buscar� los datos en el servicio de Apache
Kafka y los dejar� listos para ser procesados.

%TODO Escribir sobre resultados

%TODO Escribir sobre conclusiones

la prueba de concepto puede ser extendida implementado nuevas topolog�as de
Apache Storm para realizar procesamientos m�s complejos de los datos, utilizar
un servicio en la nube para alojar los servidores. Por otro lado, implementar
mecanismos de auto escalabilidad de cada uno de los servicios para que el
sistema crezca y decrezca dependiendo de las necesidades de procesamiento del
momento. Tambi�n, es posible implementar mecanismos de auto descubrimiento de
servicios para garantizar que al crear o borrar un nodo de alg�n servicio el
sistema continue funcionando de manera �ptima.






