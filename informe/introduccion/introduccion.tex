%%%%%%%%%%%%%%%%%%%%%%%%%%%%%%%%%%%%%%%%%%%%%%%%%%%%%%%%%%%%%%%%%%%%%%%%%%%%%%%%%
%																				%
%	TRABAJO:	Trabajo Final													%
%				Especialidad en Ingenier�a en Sistemas de Informaci�n			%
%																				%
%		Titulo:																	%
%																				%
%		Autor:	Juli�n Nonino													%
%																				%
%	Introducci�n																%	
%																				%
%	A�o: 2016																	%
%																				%
%%%%%%%%%%%%%%%%%%%%%%%%%%%%%%%%%%%%%%%%%%%%%%%%%%%%%%%%%%%%%%%%%%%%%%%%%%%%%%%%%

\chapter{Introducci�n}
	

	
\section{Objetivos}

	El objetivo principal de �ste trabajo es implementar un sistema de procesamiento
	de datos en tiempo real como prueba de concepto utilizando las �ltimas
	tecnolog�as de la industria como son Docker, Apache Zookeeper, Apache Kafka y
	Apache Storm.
	
	\subsection{Objetivos Secundarios}
	
		\begin{itemize}
		    \item Plantear un modelo de generaci�n y procesamiento de datos sencillo que
		    ayude a visualizar el funcionamiento del sistema.
		    \item Demostrar el rol de Apache Zookeeper dentro de los sistemas
		    distribuidos. Mostrar su implementaci�n a trav�s de una prueba de concepto.
		    \item Demostrar el rol de Apache Kafka y Apache Storm dentro de los sistemas
		    de procesamiento de datos en tiempo real.
		    \item Estudiar y utilizar Docker como herramienta de despliegue de los
		    componentes del sistema ayudando a la escalabilidad del sistema.
		    \item Utilizar Docker Compose como mecanismo de despliegue del
		    sistema y conexi�n de los componentes del mismo.
		\end{itemize}
