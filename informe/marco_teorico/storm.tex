%%%%%%%%%%%%%%%%%%%%%%%%%%%%%%%%%%%%%%%%%%%%%%%%%%%%%%%%%%%%%%%%%%%%%%%%%%%%%%%%%
%																				%
%	TRABAJO:	Trabajo Final													%
%				Especialidad en Ingenier�a en Sistemas de Informaci�n			%
%																				%
%		Titulo:																	%
%																				%
%		Autores:	Julian Nonino												%
%																				%
%	Capitulo sobre Apache Storm													%	
%																				%
%	A�o: 2016																	%
%																				%
%%%%%%%%%%%%%%%%%%%%%%%%%%%%%%%%%%%%%%%%%%%%%%%%%%%%%%%%%%%%%%%%%%%%%%%%%%%%%%%%%

\chapter{Apache Storm}
\label{chapter_apache_storm}




Apache Storm is an open source framework that provides massively scalable event
collection. Storm was created by Twitter and is composed of other open source
components, especially ZooKeeper for cluster management, ZeroMQ for multicast
messaging, and Kafka for queued messaging.

Storm runs in production in several deployments. Storm is in the incubator stage
of Apache standard process - current version is 0.9.1-incubating. No
commercial support is available today, though Storm is adopted more and more. In
the meantime, some Hadoop vendors such as Hortonworks are adding it to their
platform step by step. The current release of Apache Storm is a sound choice if
you are looking for a stream processing framework. If your team wants to
implement a custom application by coding without any license fees, then Storm is
worth considering. Brian Bulkowski, founder of Aerospike (a company which offers
a NoSQL database with connectors to Storm) has great introductory slides, which
let you get a feeling about how to install, develop and run Storm applications.
Storm website shows some reference use cases for stream processing at
companies such as Groupon, Twitter, Spotify, HolidayCheck, Alibaba, and others.
