%%%%%%%%%%%%%%%%%%%%%%%%%%%%%%%%%%%%%%%%%%%%%%%%%%%%%%%%%%%%%%%%%%%%%%%%%%%%%%%%%
%																				%
%	TRABAJO:	Trabajo Final													%
%				Especialidad en Ingenier�a en Sistemas de Informaci�n			%
%																				%
%		Titulo:																	%
%																				%
%		Autores:	Julian Nonino												%
%																				%
%	Capitulo sobre Apache Spark													%	
%																				%
%	A�o: 2016																	%
%																				%
%%%%%%%%%%%%%%%%%%%%%%%%%%%%%%%%%%%%%%%%%%%%%%%%%%%%%%%%%%%%%%%%%%%%%%%%%%%%%%%%%

\chapter{Apache Spark}

Spark es una herramienta de c�digo abierto desarrollada para procesar datos de
manera r�pida y f�cil. Su desarrollo comenz� en 2009 en el AMPLab de la
Universidad de Berkeley, siendo liberado su c�digo en 2010 como un proyecto
Apache.
Spark provee herramientas para procesar diversos conjuntos de datos de distinta
naturaleza (textos, grafos, etc�tera) y datos de distintas fuentes,
procesamiento de datos por lotes o procesamiento de un flujo de datos en tiempo
real.
Las aplicaciones para Spark pueden ser escritas en Java, Scala o Python y el
paquete incluye mas de 80 operadores de alto nivel para trabajar con los datos.
Adem�s de operaciones \emph{Map and Reduce} sobre Hadoop, soporta consultas SQL,
flujos de datos y \emph{Machine Learning} \cite{Penchikala2015SparkIntro}.
