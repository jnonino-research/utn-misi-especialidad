%%%%%%%%%%%%%%%%%%%%%%%%%%%%%%%%%%%%%%%%%%%%%%%%%%%%%%%%%%%%%%%%%%%%%%%%%%%%%%%%%
%																				%
%	TRABAJO:	Trabajo Final													%
%				Especialidad en Ingenier�a en Sistemas de Informaci�n			%
%																				%
%		Titulo:																	%
%																				%
%		Autores:	Juli�n Nonino												%
%																				%
%	DOCUMENTO PRINCIPAL															%	
%																				%
%	A�o: 2016																	%
%																				%
%%%%%%%%%%%%%%%%%%%%%%%%%%%%%%%%%%%%%%%%%%%%%%%%%%%%%%%%%%%%%%%%%%%%%%%%%%%%%%%%%

\documentclass[a4paper,12pt,openright,twoside]{book}

% Paquetes
	% Idioma y codificacion de caracteres
		\usepackage[spanish]{babel}
		\usepackage[latin1]{inputenc}
	% Figuras
		\usepackage{graphicx}
		\usepackage{subfigure}
		\usepackage{float} % Para posicionar imagenes donde uno quiera. Solo hay que poner la opcion [H]
	% Apendice
		\usepackage{appendix}
	%Tablas	
	%\usepackage{tabular}
	% Margenes
		\usepackage{anysize}
	% Tabla de conteido
		\usepackage[tight]{shorttoc}
	% Matematica
		\usepackage[cmex10]{amsmath}
		\usepackage{amssymb}
	% Referencias
		\usepackage[dcucite]{harvard}
		\usepackage{hyperref}
	% Colores
		\usepackage{color}
		% Definicion de colores
			\definecolor{dkgreen}{rgb}{0,0.6,0}
			\definecolor{gray}{rgb}{0.5,0.5,0.5}
			\definecolor{mauve}{rgb}{0.58,0,0.82}
			\definecolor{violeta}{RGB}{127,0,85}
	% Insertar c�digo
		\usepackage{listings}

% Margenes
	% Controla los m�rgenes {izquierda}{derecha}{arriba}{abajo}
		\marginsize{3cm}{3cm}{2.5cm}{2.5cm}

% Encabezados
	\pagestyle{headings}
		
% Documento
\begin{document}
 
	% Reeescritura de comandos
		\renewcommand{\appendixname}{Ap�ndice}
		\renewcommand{\appendixtocname}{Ap�ndice}
		\renewcommand{\tablename}{\textbf{Tabla}} 	% Para poner la palabra en mayusucula
		\renewcommand{\figurename}{\textbf{Figura}} % Para poner la palabra en mayuscula
		\renewcommand{\contentsname}{�ndice}
		\renewcommand{\listtablename}{�ndice de tablas}
		\renewcommand{\listfigurename}{�ndice de Figuras}

		\setcounter{secnumdepth}{3} % Para numerar subsubsecciones
		\setcounter{tocdepth}{3}	% Para incluir subsubsecciones en la TOC	
	% INICIO DE LA PRIMERA PARTE. Resumen ejecutivo
 		\frontmatter
 		% Portada
 			\begin{titlepage}
 				%%%%%%%%%%%%%%%%%%%%%%%%%%%%%%%%%%%%%%%%%%%%%%%%%%%%%%%%%%%%%%%%%%%%%%%%%%%%%%%%%
%																				%
%	TRABAJO:	Trabajo Final													%
%				Especialidad en Ingenier�a en Sistemas de Informaci�n			%
%																				%
%		Titulo:																	%
%																				%
%		Autores:	Julian Nonino												%
%																				%
%	Portada																		%	
%																				%
%	A�o: 2016																	%
%																				%
%%%%%%%%%%%%%%%%%%%%%%%%%%%%%%%%%%%%%%%%%%%%%%%%%%%%%%%%%%%%%%%%%%%%%%%%%%%%%%%%%

% PAGINA ANTERIOR
	\vspace*{0.15in}

	\begin{center}		
	
		\begin{LARGE}
			\textbf{Procesamiento de Datos en Tiempo Real}
		\end{LARGE}
		\\
		\vspace*{0.15cm}
		\begin{Large}
			\textbf{Conceptos y An�lisis de Herramientas}
		\end{Large}
		
		\vspace*{0.15cm}
		\rule{15cm}{0.1mm} 
		\vspace*{0.15cm}
		\begin{Large}Especializaci�n en Ingenier�a en Sistemas de
		Informaci�n\end{Large}\\
	
		\vspace*{2cm}
	
		\begin{Huge}
			Julian Nonino
		\end{Huge}
		\\
		
		\vspace*{2cm}
		
%		\begin{Large}
%			Director
%			\\
%			\vspace*{0.5cm}
%			Mart�n Miceli
%		\end{Large}
		
		\vspace*{4.5cm}
		
		\begin{figure}[H]
			\centering
			\includegraphics[width=.10\linewidth]{./portada/logo_utn}
		\end{figure}	
		
		
		\begin{large}Universidad Tecnol�gica Nacional\end{large}\\
		\vspace*{0.25cm}
		\begin{large}Facultad Regional C�rdoba\end{large}\\
		\vspace*{0.25cm}
		\begin{large}Direcci�n de Posgrado\end{large}\\
		\vspace*{0.5cm}
		C�rdoba\\
		- 2016 -
	\end{center}

% PAGINA POSTERIOR
	\newpage
	\mbox{}
	\thispagestyle{empty}
	
 				\thispagestyle{empty}
 			\end{titlepage}
 		% Resumen
 			% Resumen ejecutivo
 				%%%%%%%%%%%%%%%%%%%%%%%%%%%%%%%%%%%%%%%%%%%%%%%%%%%%%%%%%%%%%%%%%%%%%%%%%%%%%%%%%%
%																				%
%	TRABAJO:	Trabajo Final													%
%				Especialidad en Ingenier�a en Sistemas de Informaci�n			%
%																				%
%		Titulo:																	%
%																				%
%		Autores:	Julian Nonino												%
%																				%
%	Resumen Ejecutivo															%	
%																				%
%	A�o: 2016																	%
%																				%
%%%%%%%%%%%%%%%%%%%%%%%%%%%%%%%%%%%%%%%%%%%%%%%%%%%%%%%%%%%%%%%%%%%%%%%%%%%%%%%%%

\chapter*{Resumen Ejecutivo}
 			% Executive Summary
 				%%%%%%%%%%%%%%%%%%%%%%%%%%%%%%%%%%%%%%%%%%%%%%%%%%%%%%%%%%%%%%%%%%%%%%%%%%%%%%%%%%
%																				%
%	TRABAJO:	Trabajo Final													%
%				Especialidad en Ingenier�a en Sistemas de Informaci�n			%
%																				%
%		Titulo:	Procesamiento de Datos en Tiempo Real							%
%																				%
%		Autor:	Juli�n Nonino													%
%																				%
%	Executive Summary															%	
%																				%
%	A�o: 2016																	%
%																				%
%%%%%%%%%%%%%%%%%%%%%%%%%%%%%%%%%%%%%%%%%%%%%%%%%%%%%%%%%%%%%%%%%%%%%%%%%%%%%%%%%

\chapter*{Executive Summary}

 		% Tabla de contenido inicial
 			\shorttableofcontents{Tabla de contenido}{0}
 		
 	%INICIO DEL TEXTO PRINCIPAL
 		\mainmatter
 		% Introducci�n
 			\part{Introducci�n}
 				% Introducci�n
					%%%%%%%%%%%%%%%%%%%%%%%%%%%%%%%%%%%%%%%%%%%%%%%%%%%%%%%%%%%%%%%%%%%%%%%%%%%%%%%%%
%																				%
%	TRABAJO:	Trabajo Final													%
%				Especialidad en Ingenier�a en Sistemas de Informaci�n			%
%																				%
%		Titulo:																	%
%																				%
%		Autor:	Juli�n Nonino													%
%																				%
%	Introducci�n																%	
%																				%
%	A�o: 2016																	%
%																				%
%%%%%%%%%%%%%%%%%%%%%%%%%%%%%%%%%%%%%%%%%%%%%%%%%%%%%%%%%%%%%%%%%%%%%%%%%%%%%%%%%

\chapter{Introducci�n}
	
\section{Objetivos}
 		% Marco te�rico
 			\part{Marco Te�rico}
 				% Procesamiento de Datos en Tiempo Real
 					\chapter{Procesamiento de Datos en Tiempo Real}

En los �ltimos tiempos, la demanda de procesamiento de flujos continuos de datos
(data streams) se ha incrementado considerablemente. Esto se debe a que ya no es
suficiente con procesar grandes vol�menes de datos. Los datos, adem�s, deben ser
procesados r�pidamente permitiendo a los sistemas reaccionar ante los eventos lo
antes posible. Ejemplos de sistemas que necesitan �ste nivel de procesamiento
son los sistemas de detecci�n de fraude, monitoreo de recursos, comercio,
etc�tera.

\section{Big Data}

	El t�rmino Big Data, muy utilizado en la actualidad, hace referencia a lo que
	se conoce como las tres V, Volumen, Variedad y Velocidad. Con ello, se quiere
	indicar que un sistema Big Data no solo implica trabajar con grandes vol�menes
	de datos, sino que estos datos pueden ser muy variados y se deben
	procesarr�pidamente.

	\begin{figure}[H]
		\centering
		\includegraphics[width=.10\linewidth]{./marco_teorico/img/big_data_tres_v}
	\end{figure}

 				% Docker
 					%%%%%%%%%%%%%%%%%%%%%%%%%%%%%%%%%%%%%%%%%%%%%%%%%%%%%%%%%%%%%%%%%%%%%%%%%%%%%%%%%
%																				%
%	TRABAJO:	Trabajo Final													%
%				Especialidad en Ingenier�a en Sistemas de Informaci�n			%
%																				%
%		Titulo:																	%
%																				%
%		Autores:	Julian Nonino												%
%																				%
%	Capitulo sobre Docker														%	
%																				%
%	A�o: 2016																	%
%																				%
%%%%%%%%%%%%%%%%%%%%%%%%%%%%%%%%%%%%%%%%%%%%%%%%%%%%%%%%%%%%%%%%%%%%%%%%%%%%%%%%%

\chapter{Docker}
\label{chapter_docker}

Docker es un proyecto de c�digo abierto que automatiza el despliegue de
aplicaciones dentro de contenedores de software, proporcionando una capa
adicional de abstracci�n y automatizaci�n de virtualizaci�n a nivel de sistema
operativo en Linux. Docker utiliza caracter�sticas de aislamiento de recursos
del kernel de Linux, tales como cgroups y espacios de nombres (namespaces) para
permitir que \emph{contenedores} independientes se ejecuten dentro de una sola
instancia de Linux, evitando la sobrecarga de iniciar y mantener m�quinas
virtuales. \cite{WikipediaDocker}

El soporte del kernel de Linux para los espacios de nombres a�sla de vista, en
su mayor�a, una aplicaci�n del entorno operativo, incluyendo �rboles de proceso,
red, ID de usuario y sistemas de archivos montados, mientras que los cgroups del
kernel proporcionan aislamiento de recursos, incluyendo la CPU, la memoria, el
bloque de E/S y de la red. Desde la versi�n 0.9, Docker incluye la librer�a
libcontainer como su propia manera de utilizar directamente las facilidades de
virtualizaci�n que ofrece el kernel de Linux, adem�s de utilizar las interfaces
abstra�das de virtualizaci�n mediante libvirt, LXC (Linux Containers) y
systemd-nspawn. \cite{WikipediaDocker}

Docker implementa una API de alto nivel para proporcionar contenedores livianos
que ejecutan procesos de manera aislada.
Construido sobre las facilidades proporcionadas por el kernel de Linux
(principalmente cgroups y namespaces), un contenedor Docker, a diferencia de una
m�quina virtual, no requiere incluir un sistema operativo independiente. En su
lugar, se basa en las funcionalidades del kernel y utiliza el aislamiento de
recursos (CPU, la memoria, el bloque E/S, red, etc.) y namespaces separados
para aislar de vista la aplicaci�n del sistema operativo. Docker accede a la
virtualizaci�n del kernel de Linux ya sea directamente a trav�s de la biblioteca
libcontainer (disponible desde Docker 0.9), o indirectamente a trav�s de
libvirt, LXC o systemd-nspawn. \cite{WikipediaDocker}

\begin{figure}[H]
	\centering
	\includegraphics[width=1\linewidth]{./marco_teorico/img/docker/vm_vs_docker}
	\caption{Contenedor Docker (derecha) versus M�quina Virtual (izquierda)
	\cite{WhatIsDocker2016}}
\end{figure}

Mediante el uso de contenedores, los recursos pueden ser aislados, los servicios
restringidos, y se otorga a los procesos la capacidad de tener una visi�n casi
completamente privada del sistema operativo con su propio identificador de
espacio de proceso, la estructura del sistema de archivos, y las interfaces de
red. Contenedores m�ltiples comparten el mismo n�cleo, pero cada contenedor
puede ser restringido a utilizar s�lo una cantidad definida de recursos como
CPU, memoria y E/S. \cite{WikipediaDocker}

Usando Docker para crear y gestionar contenedores se puede simplificar la
creaci�n de sistemas altamente distribuidos, permitiendo m�ltiples aplicaciones,
las tareas de los trabajadores y otros procesos para funcionar de forma aut�noma
en una �nica m�quina f�sica o en varias m�quinas virtuales. Esto permite que el
despliegue de nodos se realice a medida que se dispone de recursos o cuando se
necesiten m�s nodos, lo que permite una plataforma como servicio (PaaS -
Plataform as a Service) de estilo de despliegue y ampliaci�n de los sistemas
como Apache Cassandra, MongoDB o Riak. Docker tambi�n simplifica la creaci�n y
el funcionamiento de las tareas de carga de trabajo o las colas y otros sistemas
distribuidos. \cite{WikipediaDocker}

\section{Imagenes y Contenedores\cite{GetStartedDocker2016}}
	
	Un \emph{contenedor} (container) es una version de un sistema operativo Linux,
	solo con los componentes m�s b�sicos. Una \emph{imagen} es software que se
	carga dentro del container al momento de ejecutar el comando \emph{run}.
	
\lstset{language=bash}
\begin{lstlisting}
docker run hello-world
\end{lstlisting}
	
	El comando \emph{run} recibe como par�metro requerido el nombre de la
	\emph{imagen} que se desea cargar en un \emph{contenedor}, en �ste caso
	\emph{hello-world}.
	
	Al correr dicho comando, Docker ejecuta las siguientes acciones:
	\begin{itemize}
	    \item Comprobar si existe en el sistema una imagen con el nombre
	    \emph{hello-world}.
	    \item En caso de que no exista dicha imagen en el sistema descargarla desde
	    el repositorio de im�genes configurado, por defecto es \emph{Docker Hub},
	    un repositorio propiedad de Docker donde existen miles de imagenes
	    disponibles. Es posible tener repositorios privados utilizando lo que se
	    conoce como \emph{Docker Registry}.
	    \item Cargar la imagen en el contenedor y ejecutarla.
	\end{itemize}
	
	Por otro lado, una imagen de Docker puede ejecutar un simple comando o cargar
	un complejo sistema de base de datos.
	
	Para construir una imagen de Docker, es necesario crear un archivo llamado
	\emph{Dockerfile}.
	
\lstset{language=bash}
\begin{lstlisting}
FROM ubuntu:16.04

RUN apt-get -y update

CMD["echo Hola"]
\end{lstlisting}
	
	El Dockerfile anterior buscara una imagen de Ubuntu con la etiqueta
	(\emph{tag}) \emph{16.04}. Luego ejecutar� un comando para actualizar los
	paquetes del sistema operativo y luego mostrar� el mensaje \emph{Hola}.

	El comando para construir una imagen de Docker es:

\lstset{language=bash}
\begin{lstlisting}
docker build -t miimagen .
\end{lstlisting}

	Se ejecutar� el comando \emph{build} para construir la imagen. El argumento
	\emph{-t} indica que se le pondr� la etiqueta \emph{miimagen} a la imagen y
	punto al final indica el directorio de contexto de la imagen, �sto es �til
	porque se pueden agregar archivos al container al momento de construir la
	imagen. En �ste caso, el contexto ser� el directorio donde se encuentra el
	Dockerfile.

	Luego, es posible cargar la imagen en un contenedor mediante el comando:
	
\lstset{language=bash}
\begin{lstlisting}
docker run miimagen
\end{lstlisting}

	\subsection{Crear nuevas etiquetas}
	
		Para ponerle una nueva etiqueta a una imagen, primero debemos encontrar el
		n�mero de identificaci�n de la imagen. �sto se hace corriendo el comando:

\lstset{language=bash}
\begin{lstlisting}
docker images
\end{lstlisting}

	El comando anterior, mostrar� una lista de las im�genes existentes en el
	sistema mostrando la �ltima etiqueta de la misma, el n�mero de identificaci�n,
	la fecha de creaci�n y el tama�o de la imagen.
	
	Luego, para aplicarle una nueva etiqueta, se ejecuta el comando:
	
\lstset{language=bash}
\begin{lstlisting}
docker tag <IMAGE_ID> <NUEVA_ETIQUETA>
\end{lstlisting}	
	
\section{Docker Compose\cite{DockerComposeDocumentation}}

	Docker Compose es una herramienta que permite correr un sistema formado por
	m�ltiples contenedores. Para ello, se debe crear un archivo \emph{.yml} en el
	que se definan los servicios con los que va a contar la aplicaci�n. Cada
	servicio estar� formado por un contenedor corriendo una imagen de Docker.
	
	Para cada servicio pueden definirse nombres, puertos expuestos, conexiones de
	red, etc�tera, luego, con los siguientes comandos se puede operar con el
	sistema.
	
	Para una lista completa de los comandos de Docker Compose, acceder a
	\href{https://docs.docker.com/compose/reference/}{Docker Compose Command-Line
	Reference}\footnote{https://docs.docker.com/compose/reference/}.

 				% Apache Kafka
 					%%%%%%%%%%%%%%%%%%%%%%%%%%%%%%%%%%%%%%%%%%%%%%%%%%%%%%%%%%%%%%%%%%%%%%%%%%%%%%%%%
%																				%
%	TRABAJO:	Trabajo Final													%
%				Especialidad en Ingenier�a en Sistemas de Informaci�n			%
%																				%
%		Titulo:																	%
%																				%
%		Autores:	Julian Nonino												%
%																				%
%	Capitulo sobre Apache Kafka													%	
%																				%
%	A�o: 2016																	%
%																				%
%%%%%%%%%%%%%%%%%%%%%%%%%%%%%%%%%%%%%%%%%%%%%%%%%%%%%%%%%%%%%%%%%%%%%%%%%%%%%%%%%

\chapter{Apache Kafka}
\label{chapter_apache_kafka}

Kafka es un sistema de mensajes distribuido, particionado y con
replicaci�n\cite{ApacheKafka090}.

\begin{itemize}
    \item Kafka mantiene los mensajes agrupados en categor�as llamadas
    \emph{topics.}
	\item Los productores de mensajes se llaman \emph{producers}.
	\item Los consumidores de mensajes se llaman \emph{consumers}.
	\item Kafka corre en un cluster formado por uno o mas servidores. cada uno de
	ellos es llamado \emph{broker}.
\end{itemize}

\begin{figure}[H]
	\centering
	\includegraphics[width=.5\linewidth]{./marco_teorico/img/kafka/high_level_arch}
	\caption{Kafka, arquitectura de alto nivel\cite{ApacheKafka090}}
\end{figure}

\section{Topics}

	Los \emph{topics} de Kafka son categor�as de mensajes para los cuales Kafka
	mantiene registros particionados.
	
	Cada partici�n es una secuencia ordenada e inmutable de mensajes. El n�mero de
	orden de cada mensaje es llamado \emph{offset} e identifica univocamente a cada
	mensaje de la partici�n.
	
	Kafka mantiene los mensajes publicados por un per�odo de tiempo configurable,
	sin importar si fueron consumidos o no por alg�n proceso \emph{consumer}. Cada
	consumidor se encarga de mantener el \emph{offset} y tiene libertad para ir
	hacia atr�s y hacia adelante en los mensajes publicados para procesarlos.
	
	El tener los mensajes de un \emph{topic} particionados permite separar el
	\emph{topic} en varios servidores. �sto permite manejar grandes volumenes de
	datos y adem�s otorogar un nivel superior de paralelismo.
	
	\begin{figure}[H]
		\centering
		\includegraphics[width=.5\linewidth]{./marco_teorico/img/kafka/kafka_topics}
		\caption{Topics en Kafka\cite{ApacheKafka090}}
	\end{figure}
	
	Cada partici�n est� formada por un l�der (\emph{leader}) que se encuentra en
	uno de los servidores y por cero o m�s seguidores (\emph{followers}) que
	replican al l�der todo el tiempo en servidores distintos. Si el l�der falla,
	alguno de los seguidores se convertir� en el nuevo l�der garantizando que el
	sistema siga operando. la configuraci�n ideal es que cada servidor sea l�der de
	alguna partici�n y seguidor de las otras.
	
\section{Productores}
	
	Los productores en Kafka son programas encargados de publicar datos en los
	\emph{topics}. El productor decide, para cada mensaje, el topic y la partici�n
	en el cual publicarlo. Generalmente la partici�n es elegida siguiendo un
	esquema \emph{round-robin} para lograr un �ptimo balance de carga entre
	particiones, pero se puede utilizar cualquier l�gica.
	
\section{Consumidores}

	Los sitemas de mensajer�a pueden ser clasificados en dos categor�as,
	\emph{cola de mensajes} o \emph{publicaci�n-subscripci�n}. En el primero, los
	mensajes son encolados y cada mensaje es dirigido hacia alguno de los
	consumidores. En el segundo, cada mensaje es transmitido a todos los
	consumidores. kafka maneja ambos mundos con lo que se conoce como grupos de
	consumidores (\emph{consumer groups}).
	
	Cada consumidor debe ubicarse dentro de alguno de los grupos de consumidores y
	cuando un mensajes es publicado en un \emph{topic}, el mensaje es transmitido a
	un �nico consumidor de cada uno de los grupos de consumidores.
	
	\begin{figure}[H]
		\centering
		\includegraphics[width=.5\linewidth]{./marco_teorico/img/kafka/kafka_consumers}
		\caption{Grupos de Consumidores\cite{ApacheKafka090}}
	\end{figure}	
	
	Si todos los consumidores se encuentran en el mismo grupo, el sistema funciona
	como una cola de mensajes distribuyendo la carga entre cada uno de los
	consumidores.
	
	Si todos los consumidores se encuentran en distintos grupos, el sistema
	funciona como un sistema publicaci�n-subscripci�n y todos los mensajes son
	transmitidos a todos los consumidores.
	
				% Apache Storm
 					%%%%%%%%%%%%%%%%%%%%%%%%%%%%%%%%%%%%%%%%%%%%%%%%%%%%%%%%%%%%%%%%%%%%%%%%%%%%%%%%%
%																				%
%	TRABAJO:	Trabajo Final													%
%				Especialidad en Ingenier�a en Sistemas de Informaci�n			%
%																				%
%		Titulo:																	%
%																				%
%		Autores:	Julian Nonino												%
%																				%
%	Capitulo sobre Apache Storm													%	
%																				%
%	A�o: 2016																	%
%																				%
%%%%%%%%%%%%%%%%%%%%%%%%%%%%%%%%%%%%%%%%%%%%%%%%%%%%%%%%%%%%%%%%%%%%%%%%%%%%%%%%%

\chapter{Apache Storm}





Apache Storm is an open source framework that provides massively scalable event
collection. Storm was created by Twitter and is composed of other open source
components, especially ZooKeeper for cluster management, ZeroMQ for multicast
messaging, and Kafka for queued messaging.

Storm runs in production in several deployments. Storm is in the incubator stage
of Apache standard process - current version is 0.9.1-incubating. No
commercial support is available today, though Storm is adopted more and more. In
the meantime, some Hadoop vendors such as Hortonworks are adding it to their
platform step by step. The current release of Apache Storm is a sound choice if
you are looking for a stream processing framework. If your team wants to
implement a custom application by coding without any license fees, then Storm is
worth considering. Brian Bulkowski, founder of Aerospike (a company which offers
a NoSQL database with connectors to Storm) has great introductory slides, which
let you get a feeling about how to install, develop and run Storm applications.
Storm website shows some reference use cases for stream processing at
companies such as Groupon, Twitter, Spotify, HolidayCheck, Alibaba, and others.

 				% Apache Spark
 					%%%%%%%%%%%%%%%%%%%%%%%%%%%%%%%%%%%%%%%%%%%%%%%%%%%%%%%%%%%%%%%%%%%%%%%%%%%%%%%%%
%																				%
%	TRABAJO:	Trabajo Final													%
%				Especialidad en Ingenier�a en Sistemas de Informaci�n			%
%																				%
%		Titulo:																	%
%																				%
%		Autores:	Julian Nonino												%
%																				%
%	Capitulo sobre Apache Spark													%	
%																				%
%	A�o: 2016																	%
%																				%
%%%%%%%%%%%%%%%%%%%%%%%%%%%%%%%%%%%%%%%%%%%%%%%%%%%%%%%%%%%%%%%%%%%%%%%%%%%%%%%%%

\chapter{Apache Spark}	
 				% Apache Flink
 					%%%%%%%%%%%%%%%%%%%%%%%%%%%%%%%%%%%%%%%%%%%%%%%%%%%%%%%%%%%%%%%%%%%%%%%%%%%%%%%%%
%																				%
%	TRABAJO:	Trabajo Final													%
%				Especialidad en Ingenier�a en Sistemas de Informaci�n			%
%																				%
%		Titulo:																	%
%																				%
%		Autores:	Julian Nonino												%
%																				%
%	Capitulo sobre Apache Flink													%	
%																				%
%	A�o: 2016																	%
%																				%
%%%%%%%%%%%%%%%%%%%%%%%%%%%%%%%%%%%%%%%%%%%%%%%%%%%%%%%%%%%%%%%%%%%%%%%%%%%%%%%%%

\chapter{Apache Flink}
\label{chapter_apache_flink}



\section{Conceptos\cite{ApacheFlinkDocs}}

	Los \emph{programas Flink} son programas comunes que implementan
	transformaciones en colecciones distribuidas, por ejemplo, filtrado,
	correspondencia, actualizaci�n de estado, uniones, agrupamientos, agregaciones,
	etc�tera. �stas colecciones se forman a partir de las fuentes de datos
	(\emph{sources}). Dichas fuentes se forman leyendo archivos, conectando Flink a
	un servidor de mensajes como Apache Kafka \ref{chapter_apache_kafka} o mediante
	colecciones definidas localmente.
	
	Los resultados de la ejecuci�n de un programa Flink son devueltos mediante el
	uso de receptores de datos \emph{sinks}. �stos receptores pueden consistir en
	escritura de archivos, impresi�n en la consola de ejecuci�n, etc�tera.
	
	Los programas Flink pueden correr localmente (standalone), embebidos en otros
	programas o en clusters.
	
	Dependiendo del tipo de fuente de datos (\emph{source}), es decir, acotados o
	no acotados, el programa Flink deber� realizar una ejecuci�n por lotes
	(\emph{batch}) o una ejecuci�n en tiempo real sobre el flujo de datos
	(\emph{straming}). Para el primer caso, se deber� utilizar la
	\textbf{\emph{DataSet API}} y para el segundo caso la \textbf{\emph{DataStream
	API}}.
	
	Los bloques b�sicos de un programa Flink son los flujos de datos (streams) y
	las transformaciones (operaciones).

	Al ejecutarse, un programa Flink se corresponde con lo que se conoce como
	\emph{Streaming Dataflow}. Cada \emph{Dataflow}, comienza con una o m�s fuentes
	de datos (\emph{sources}) y termina en uno o m�s receptores de datos
	(\emph{sinks}).
	En la mayor�a de los casos, existe una correspondencia uno a uno entre las
	transformaciones especificadas en el programa y las operaciones del
	\emph{Dataflow} pero puede ocurrir que una transformaci�n est� formada por mas
	de un operador de transformaci�n.
	
	\begin{figure}[H]
		\centering
		\includegraphics[width=1\linewidth]{./marco_teorico/img/flink/building_blocks}
		\caption{Bloques Fundamentales de un Programa Flink\cite{ApacheFlinkDocs}}
	\end{figure}

\subsection{Usar Flink}
	
		Para escribir un programa Flink, se deben incluir las libre�as Flink en el
		proyecto, en el caso de Maven, �sto se logra insertando las siguientes l�neas
		en el pom.xml del proyecto.
		
\lstset{language=XML}
\begin{lstlisting}
<dependency>
	<groupId>org.apache.flink</groupId>
	<artifactId>flink-core_2.11</artifactId>
	<version>1.0.3</version>
</dependency>
<dependency>
	<groupId>org.apache.flink</groupId>
	<artifactId>flink-java_2.11</artifactId>
	<version>1.0.3</version>
</dependency>
<dependency>
	<groupId>org.apache.flink</groupId>
	<artifactId>flink-clients_2.11</artifactId>
	<version>1.0.3</version>
</dependency>
<dependency>
	<groupId>org.apache.flink</groupId>
	<artifactId>flink-streaming-java_2.11</artifactId>
	<version>1.0.3</version>
</dependency>
<dependency>
	<groupId>org.apache.flink</groupId>
	<artifactId>flink-connector-kafka-0.9_2.11</artifactId>
	<version>1.0.3</version>
</dependency>
\end{lstlisting}
	
	\subsection{DataSet y DataStreams\cite{ApacheFlinkDocs}}		

		Para representar datos en un programa Flink existen dos tipos de clases
		DataSet y DataStream. Se puede considerar que son colecciones inmutables de
		datos que pueden contener duplicados. En el caso del DataSet la cantidad de
		datos es finita, mientras que en un DataStream pueden ser ilimitados.
		
		�stas colecciones son diferentes a las Java en el sentido de que son
		inmutables, una vez creadas no pueden a�adirse ni removerse elementos. Tampoco
		es posible inspeccionar los elementos contenidos dentro de la colecci�n.
		
		Como se menciono anteriormente, una colecci�n DataSet o DataStream es creada
		en el momento en que se a�ade una fuente de datos \emph{source} y nuevas
		colecciones son creadas cada vez que una operaci�n de transformaci�n es
		ejecutada.

	\subsection{Evaluaci�n Postergada}
	
		Al ejecutar un programa Flink, el m�todo \emph{main} es ejecutado pero la
		carga de datos y las transformaciones no ocurren directamente. Cada operaci�n
		es creada y a�adida a un plan de ejecuci�n del programa. Las operaciones, son
		ejecutadas cuando son exlicitamente disparada mediante el llamado del m�todo
		\emph{execute()} sobre el entorno de ejecuci�n \cite{ApacheFlinkDocs}.
	

Flink DataStream API Programming Guide (Source: https://ci.apache.org/projects/flink/flink-docs-release-1.0/apis/streaming/index.html)

DataStream programs in Flink are regular programs that implement transformations
on data streams (e.g., filtering, updating state, defining windows,
aggregating). The data streams are initially created from various sources (e.g.,
message queues, socket streams, files). Results are returned via sinks, which
may for example write the data to files, or to standard output (for example the
command line terminal). Flink programs run in a variety of contexts, standalone,
or embedded in other programs. The execution can happen in a local JVM, or on
clusters of many machines.


 		% Desarrollo
 			\part{Desarrollo}
 				% Desarrollo e Implementaci�n
 					%\input{./desarrollo/diseno_implementacion/chap_diseno_implementacion}
 				% Implementaci�n en FPGA
 					%\input{./desarrollo/implementacion_FPGA/chap_implementacion_fpga}
 		% Resultados y Conclusiones			
 			\part{Resultados y Conclusiones}
 				%Resultados obtenidos
 					%\input{./resultados_conclusiones/resultados/chap_resultados}
 				% Conclusiones
 					%\input{./resultados_conclusiones/chap_conclusiones}
 				% Trabajo futuro
 					%\input{./resultados_conclusiones/chap_trabajo_futuro}	
 	
	% APENDICES
		\appendix
		\part{Ap�ndices}
			% Nuevo proyecto en Xilinx XPS
				%\input{./apendices/nuevos_proyectos_XPS/nuevos_proyectos_XPS}
			% Creacion de un nuevo IP core	
				%\input{./apendices/creacion_IP_core/creacion_IP_core}
			% Interrupciones en los IP core
				%\input{./apendices/interrupciones/interrupciones}
			% Gesti�n de la metodolog�a de desarrollo
				%%%%%%%%%%%%%%%%%%%%%%%%%%%%%%%%%%%%%%
%									%
%	Copyright 2014 - Julian Nonino	%								
%									%
%%%%%%%%%%%%%%%%%%%%%%%%%%%%%%%%%%%%%

\section{Metodolog�a de desarrollo}

	El supuesto de �sta tesis consiste en que es posible dise�ar e implementar un
	herramienta de ense�anza en l�nea que permita a los estudiantes relacionar
	t�cnicas de estimaci�n y planeamiento utilizadas en Scrum con pr�cticas
	tradicionales de estimaci�n utilizando planeamiento de escenarios y generaci�n
	autom�tica de datos para pron�stico.

	Basado en el supuesto anterior, la primera etapa de la tesis consistir� en la
	recolecci�n y an�lisis de material sobre el tema. Se buscar�n libros, art�culos
	y presentaciones de congresos y conferencias, \emph{white papers}, art�culos de
	revistas especializadas, adem�s de blog, sitios de noticias, foros, de opini�n,
	etc�tera.

	Los puntos m�s importantes para la realizaci�n de la Revisi�n Sistem�tica de la
	Literatura son:
	\begin{itemize}
		\item Metodolog�as �giles (Agile). 
		\item Scrum.
		\item Pr�cticas de estimaci�n y planeamiento en Scrum.
		\item Herramientas existentes para soporte de los procesos de estimaci�n y
		planeamiento en Scrum.
		\item Historias de usuario, puntos de historia, horas ideales, velocidad.
		\item Proyectos de alcance fijo, duraci�n fija o precio fijo.
		\item Estimaci�n y planeamiento para m�ltiples equipos en Scrum.
		\item Simulaci�n de Monte Carlo.
		\item Simulaci�n de escenarios alternativos.
		\item Ense�anza de las pr�cticas de estimaci�n y planeamiento utilizadas en
		Scrum a alumnos de novel universitario.
	\end{itemize}

	Analizada la bibliograf�a, se deber�n determinar los requerimientos de la
	herramienta y se deber� elaborar un dise�o que contemple las pr�cticas de
	estimaci�n y planeamiento utilizadas en Scrum, combinadas con pr�cticas
	avanzadas pensando en el �mbito universitario, profesores y alumnos, como
	destinatario de la herramienta.

	Luego, se comenzar� la construcci�n del primer prototipo de la herramienta en
	Microsoft Excel pensando la simplicidad de uso de la misma y en la claridad de
	los conceptos utilizados para que sea �til en el �mbito universitario. Dicha
	herramienta podr�a ser probada por alumnos durante los cursos de Ingenier�a de
	Software y Gesti�n de la Calidad del Software pertenecientes a la carrera
	Ingenier�a en Computaci�n de la Universidad Nacional de C�rdoba. Se deber�n
	definir m�tricas para la evaluaci�n del impacto que produzca la herramienta
	sobre los alumnos e identificar falencias, errores y mejoras para continuar el
	desarrollo.
	
	Los resultados obtenidos deber�n ser analizados detenidamente con el fin de
	identificar posibles correcciones y mejoras en los requerimientos y dise�os
	originales, considerando que la herramienta deber� ser desarrollada como
	aplicaci�n web disponible en l�nea.

	Corregidos y mejorados los requerimientos originales e identificados nuevos
	requerimientos y conceptos que deben ser implementados en la segunda versi�n de
	la herramienta, se comenzar� con la construcci�n de la misma.
	
	Para comenzar con �sta etapa se deben analizar las distintas alternativas en
	lenguajes de programaci�n, frameworks y otras herramientas utilizadas en el
	mercado para el desarrollo de aplicaciones web. Adem�s, se deben analizar
	alternativas para hacer que la aplicaci�n web desarrollada pueda ser puesta en
	l�nea.
	
	Estando construida la herramienta, puede ser probada nuevamente en el �mbito
	universitario, como se mencion� anteriormente. Dicha prueba generar� resultados
	y estad�sticas de uso, ser�n encontrados errores y mejoras y, ser�n
	identificados fortalezas y beneficios generados por la existencia de la
	herramienta. Los datos recolectados ser�n analizados, contrastados con la
	hip�tesis y objetivos planteados al comienzo del desarrollo y luego plasmados
	en el informe de tesis con las respectivas conclusiones.
	
	\newpage

			% C�digos en Verilog de los IP cores
				%\input{./apendices/codigos_ipcores/codigos_ipcores}
			% C�digos en C de los programas ejecutados
				%\input{./apendices/codigos_programas/codigos_programas}
	
	%INICIO DE LA PARTE FINAL. Bibliografia e indices
		\backmatter
		% Referencias
			\addcontentsline{toc}{chapter}{Bibliograf�a}
			\bibliographystyle{dcu}
			\bibliography{referencias}		
		% Indices
			% Indice de contenido
				\addcontentsline{toc}{chapter}{�ndice de contenido} % para que aparezca en el indice de contenidos
				\tableofcontents
			% �ndice de Figuras
				\cleardoublepage
				\addcontentsline{toc}{chapter}{�ndice de Figuras} % para que aparezca en el indice de contenidos
				\listoffigures % indice de Figuras

\end{document}
