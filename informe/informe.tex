%%%%%%%%%%%%%%%%%%%%%%%%%%%%%%%%%%%%%%%%%%%%%%%%%%%%%%%%%%%%%%%%%%%%%%%%%%%%%%%%%%%%%
%																					%
%	TRABAJO: Proyecto Integrador													%
%																					%
%		Titulo: 	Desarrollo de IP cores con procesamiento de Redes de Petri 		%
%					Temporales para sistemas multicore en FPGA						%
%																					%
%		Autores:	Juli�n Nonino													%
%					Carlos Renzo Pisetta											%
%		Director:	Orlando Micolini												%
%																					%
%	DOCUMENTO PRINCIPAL																%	
%	Archivo: informe.tex															%
%																					%
%%%%%%%%%%%%%%%%%%%%%%%%%%%%%%%%%%%%%%%%%%%%%%%%%%%%%%%%%%%%%%%%%%%%%%%%%%%%%%%%%%%%%

\documentclass[a4paper,12pt,openright,twoside]{book}

% Paquetes
	% Idioma y codificacion de caracteres
		\usepackage[spanish]{babel}
		\usepackage[latin1]{inputenc}
	% Figuras
		\usepackage{graphicx}
		\usepackage{subfigure}
		\usepackage{float} % Para posicionar imagenes donde uno quiera. Solo hay que poner la opcion [H]
	% Apendice
		\usepackage{appendix}
	%Tablas	
	%\usepackage{tabular}
	% Margenes
		\usepackage{anysize}
	% Tabla de conteido
		\usepackage[tight]{shorttoc}
	% Matematica
		\usepackage[cmex10]{amsmath}
		\usepackage{amssymb}
	% Referencias
		\usepackage[dcucite]{harvard}
		\usepackage{hyperref}
	% Colores
		\usepackage{color}
		% Definicion de colores
			\definecolor{dkgreen}{rgb}{0,0.6,0}
			\definecolor{gray}{rgb}{0.5,0.5,0.5}
			\definecolor{mauve}{rgb}{0.58,0,0.82}
			\definecolor{violeta}{RGB}{127,0,85}
	% Insertar c�digo
		\usepackage{listings}

% Margenes
	% Controla los m�rgenes {izquierda}{derecha}{arriba}{abajo}
		\marginsize{3cm}{3cm}{2.5cm}{2.5cm}

% Encabezados
	\pagestyle{headings}
		
% Documento
\begin{document}
 
	% Reeescritura de comandos
		\renewcommand{\appendixname}{Ap�ndice}
		\renewcommand{\appendixtocname}{Ap�ndice}
		\renewcommand{\tablename}{\textbf{Tabla}} 	% Para poner la palabra en mayusucula
		\renewcommand{\figurename}{\textbf{Figura}} % Para poner la palabra en mayuscula
		\renewcommand{\contentsname}{�ndice}
		\renewcommand{\listtablename}{�ndice de tablas}
		\renewcommand{\listfigurename}{�ndice de Figuras}

		\setcounter{secnumdepth}{3} % Para numerar subsubsecciones
		\setcounter{tocdepth}{3}	% Para incluir subsubsecciones en la TOC	
	% INICIO DE LA PRIMERA PARTE. Resumen ejecutivo
 		\frontmatter
 		% Portada
 			\begin{titlepage}
 				%%%%%%%%%%%%%%%%%%%%%%%%%%%%%%%%%%%%%%%%%%%%%%%%%%%%%%%%%%%%%%%%%%%%%%%%%%%%%%%%%
%																				%
%	TRABAJO:	Trabajo Final													%
%				Especialidad en Ingenier�a en Sistemas de Informaci�n			%
%																				%
%		Titulo:																	%
%																				%
%		Autores:	Julian Nonino												%
%																				%
%	Portada																		%	
%																				%
%	A�o: 2016																	%
%																				%
%%%%%%%%%%%%%%%%%%%%%%%%%%%%%%%%%%%%%%%%%%%%%%%%%%%%%%%%%%%%%%%%%%%%%%%%%%%%%%%%%

% PAGINA ANTERIOR
	\vspace*{0.15in}

	\begin{center}		
	
		\begin{LARGE}
			\textbf{Procesamiento de Datos en Tiempo Real}
		\end{LARGE}
		\\
		\vspace*{0.15cm}
		\begin{Large}
			\textbf{Conceptos y An�lisis de Herramientas}
		\end{Large}
		
		\vspace*{0.15cm}
		\rule{15cm}{0.1mm} 
		\vspace*{0.15cm}
		\begin{Large}Especializaci�n en Ingenier�a en Sistemas de
		Informaci�n\end{Large}\\
	
		\vspace*{2cm}
	
		\begin{Huge}
			Julian Nonino
		\end{Huge}
		\\
		
		\vspace*{2cm}
		
%		\begin{Large}
%			Director
%			\\
%			\vspace*{0.5cm}
%			Mart�n Miceli
%		\end{Large}
		
		\vspace*{4.5cm}
		
		\begin{figure}[H]
			\centering
			\includegraphics[width=.10\linewidth]{./portada/logo_utn}
		\end{figure}	
		
		
		\begin{large}Universidad Tecnol�gica Nacional\end{large}\\
		\vspace*{0.25cm}
		\begin{large}Facultad Regional C�rdoba\end{large}\\
		\vspace*{0.25cm}
		\begin{large}Direcci�n de Posgrado\end{large}\\
		\vspace*{0.5cm}
		C�rdoba\\
		- 2016 -
	\end{center}

% PAGINA POSTERIOR
	\newpage
	\mbox{}
	\thispagestyle{empty}
	
 				\thispagestyle{empty}
 			\end{titlepage}
 		% Resumen
 			% Resumen ejecutivo
 				%%%%%%%%%%%%%%%%%%%%%%%%%%%%%%%%%%%%%%%%%%%%%%%%%%%%%%%%%%%%%%%%%%%%%%%%%%%%%%%%%%
%																				%
%	TRABAJO:	Trabajo Final													%
%				Especialidad en Ingenier�a en Sistemas de Informaci�n			%
%																				%
%		Titulo:																	%
%																				%
%		Autores:	Julian Nonino												%
%																				%
%	Resumen Ejecutivo															%	
%																				%
%	A�o: 2016																	%
%																				%
%%%%%%%%%%%%%%%%%%%%%%%%%%%%%%%%%%%%%%%%%%%%%%%%%%%%%%%%%%%%%%%%%%%%%%%%%%%%%%%%%

\chapter*{Resumen Ejecutivo}
 			% Executive Summary
 				%%%%%%%%%%%%%%%%%%%%%%%%%%%%%%%%%%%%%%%%%%%%%%%%%%%%%%%%%%%%%%%%%%%%%%%%%%%%%%%%%%
%																				%
%	TRABAJO:	Trabajo Final													%
%				Especialidad en Ingenier�a en Sistemas de Informaci�n			%
%																				%
%		Titulo:	Procesamiento de Datos en Tiempo Real							%
%																				%
%		Autor:	Juli�n Nonino													%
%																				%
%	Executive Summary															%	
%																				%
%	A�o: 2016																	%
%																				%
%%%%%%%%%%%%%%%%%%%%%%%%%%%%%%%%%%%%%%%%%%%%%%%%%%%%%%%%%%%%%%%%%%%%%%%%%%%%%%%%%

\chapter*{Executive Summary}

 		% Tabla de contenido inicial
 			\shorttableofcontents{Tabla de contenido}{0}
 		
 	%INICIO DEL TEXTO PRINCIPAL
 		\mainmatter
 		% Introducci�n
 			\part{Introducci�n}
 				% Introducci�n
					%%%%%%%%%%%%%%%%%%%%%%%%%%%%%%%%%%%%%%%%%%%%%%%%%%%%%%%%%%%%%%%%%%%%%%%%%%%%%%%%%%
%																				%
%	TRABAJO:	Trabajo Final													%
%				Especialidad en Ingenier�a en Sistemas de Informaci�n			%
%																				%
%		Titulo:																	%
%																				%
%		Autor:	Juli�n Nonino													%
%																				%
%	Introducci�n																%	
%																				%
%	A�o: 2016																	%
%																				%
%%%%%%%%%%%%%%%%%%%%%%%%%%%%%%%%%%%%%%%%%%%%%%%%%%%%%%%%%%%%%%%%%%%%%%%%%%%%%%%%%

\chapter{Introducci�n}
	
\section{Objetivos}
 				% Metodolog�a de trabajo	
 					%\input{./introduccion/intro_metodologia}
 		% Marco te�rico
 			\part{Marco Te�rico}
 				% Redes de Petri
 					%\input{./marco_teorico/redes_de_petri/chap_redes_de_petri}
 				% Concurrencia y sincronizacion
 					%\input{./marco_teorico/concurrencia/chap_concurrencia}
 				% FPGA - IP core - HDL
 					%\input{./marco_teorico/FPGA_IP_HDL/chap_fpga_ip_hdl}
 		% Desarrollo
 			\part{Desarrollo}
 				% Desarrollo e Implementaci�n
 					%\input{./desarrollo/diseno_implementacion/chap_diseno_implementacion}
 				% Implementaci�n en FPGA
 					%\input{./desarrollo/implementacion_FPGA/chap_implementacion_fpga}
 		% Resultados y Conclusiones			
 			\part{Resultados y Conclusiones}
 				%Resultados obtenidos
 					%\input{./resultados_conclusiones/resultados/chap_resultados}
 				% Conclusiones
 					%\input{./resultados_conclusiones/chap_conclusiones}
 				% Trabajo futuro
 					%\input{./resultados_conclusiones/chap_trabajo_futuro}	
 	
	% APENDICES
		\appendix
		\part{Ap�ndices}
			% Nuevo proyecto en Xilinx XPS
				%\input{./apendices/nuevos_proyectos_XPS/nuevos_proyectos_XPS}
			% Creacion de un nuevo IP core	
				%\input{./apendices/creacion_IP_core/creacion_IP_core}
			% Interrupciones en los IP core
				%\input{./apendices/interrupciones/interrupciones}
			% Gesti�n de la metodolog�a de desarrollo
				%%%%%%%%%%%%%%%%%%%%%%%%%%%%%%%%%%%%%%
%									%
%	Copyright 2014 - Julian Nonino	%								
%									%
%%%%%%%%%%%%%%%%%%%%%%%%%%%%%%%%%%%%%

\section{Metodolog�a de desarrollo}

	El supuesto de �sta tesis consiste en que es posible dise�ar e implementar un
	herramienta de ense�anza en l�nea que permita a los estudiantes relacionar
	t�cnicas de estimaci�n y planeamiento utilizadas en Scrum con pr�cticas
	tradicionales de estimaci�n utilizando planeamiento de escenarios y generaci�n
	autom�tica de datos para pron�stico.

	Basado en el supuesto anterior, la primera etapa de la tesis consistir� en la
	recolecci�n y an�lisis de material sobre el tema. Se buscar�n libros, art�culos
	y presentaciones de congresos y conferencias, \emph{white papers}, art�culos de
	revistas especializadas, adem�s de blog, sitios de noticias, foros, de opini�n,
	etc�tera.

	Los puntos m�s importantes para la realizaci�n de la Revisi�n Sistem�tica de la
	Literatura son:
	\begin{itemize}
		\item Metodolog�as �giles (Agile). 
		\item Scrum.
		\item Pr�cticas de estimaci�n y planeamiento en Scrum.
		\item Herramientas existentes para soporte de los procesos de estimaci�n y
		planeamiento en Scrum.
		\item Historias de usuario, puntos de historia, horas ideales, velocidad.
		\item Proyectos de alcance fijo, duraci�n fija o precio fijo.
		\item Estimaci�n y planeamiento para m�ltiples equipos en Scrum.
		\item Simulaci�n de Monte Carlo.
		\item Simulaci�n de escenarios alternativos.
		\item Ense�anza de las pr�cticas de estimaci�n y planeamiento utilizadas en
		Scrum a alumnos de novel universitario.
	\end{itemize}

	Analizada la bibliograf�a, se deber�n determinar los requerimientos de la
	herramienta y se deber� elaborar un dise�o que contemple las pr�cticas de
	estimaci�n y planeamiento utilizadas en Scrum, combinadas con pr�cticas
	avanzadas pensando en el �mbito universitario, profesores y alumnos, como
	destinatario de la herramienta.

	Luego, se comenzar� la construcci�n del primer prototipo de la herramienta en
	Microsoft Excel pensando la simplicidad de uso de la misma y en la claridad de
	los conceptos utilizados para que sea �til en el �mbito universitario. Dicha
	herramienta podr�a ser probada por alumnos durante los cursos de Ingenier�a de
	Software y Gesti�n de la Calidad del Software pertenecientes a la carrera
	Ingenier�a en Computaci�n de la Universidad Nacional de C�rdoba. Se deber�n
	definir m�tricas para la evaluaci�n del impacto que produzca la herramienta
	sobre los alumnos e identificar falencias, errores y mejoras para continuar el
	desarrollo.
	
	Los resultados obtenidos deber�n ser analizados detenidamente con el fin de
	identificar posibles correcciones y mejoras en los requerimientos y dise�os
	originales, considerando que la herramienta deber� ser desarrollada como
	aplicaci�n web disponible en l�nea.

	Corregidos y mejorados los requerimientos originales e identificados nuevos
	requerimientos y conceptos que deben ser implementados en la segunda versi�n de
	la herramienta, se comenzar� con la construcci�n de la misma.
	
	Para comenzar con �sta etapa se deben analizar las distintas alternativas en
	lenguajes de programaci�n, frameworks y otras herramientas utilizadas en el
	mercado para el desarrollo de aplicaciones web. Adem�s, se deben analizar
	alternativas para hacer que la aplicaci�n web desarrollada pueda ser puesta en
	l�nea.
	
	Estando construida la herramienta, puede ser probada nuevamente en el �mbito
	universitario, como se mencion� anteriormente. Dicha prueba generar� resultados
	y estad�sticas de uso, ser�n encontrados errores y mejoras y, ser�n
	identificados fortalezas y beneficios generados por la existencia de la
	herramienta. Los datos recolectados ser�n analizados, contrastados con la
	hip�tesis y objetivos planteados al comienzo del desarrollo y luego plasmados
	en el informe de tesis con las respectivas conclusiones.
	
	\newpage

			% C�digos en Verilog de los IP cores
				%\input{./apendices/codigos_ipcores/codigos_ipcores}
			% C�digos en C de los programas ejecutados
				%\input{./apendices/codigos_programas/codigos_programas}
	
	%INICIO DE LA PARTE FINAL. Bibliografia e indices
		\backmatter
		% Referencias
			\addcontentsline{toc}{chapter}{Bibliograf�a}
			\bibliographystyle{dcu}
			\bibliography{referencias}		
		% Indices
			% Indice de contenido
				\addcontentsline{toc}{chapter}{�ndice de contenido} % para que aparezca en el indice de contenidos
				\tableofcontents
			% �ndice de Figuras
				\cleardoublepage
				\addcontentsline{toc}{chapter}{�ndice de Figuras} % para que aparezca en el indice de contenidos
				\listoffigures % indice de Figuras

\end{document}
